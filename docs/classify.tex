\documentclass{article}

\usepackage[utf8]{inputenc}
\usepackage{hyperref}
\usepackage{listings}


\title{Como classificar questões}
\author{Pedro Delfino}
\date{Janeiro 2018}

\begin{document}

\maketitle

\section{Apresentação}

A ideia desse documento é explicitar o tipo de raciocínio que foi
usado durante a classificação manual das questões do
\href{https://github.com/own-pt/oab-exams/tree/master/official/raw}{repositório}. Esse
documento será usado como insumo para a construção de um algoritmo de
\textit{Machine Learning}, o que tornaria a classificação das questões
semi-automática, precisando apenas de uma supervisão.

No momento, o repositório conta com 15 exames que foram totalmente
classificados por áreas. Atualmente, para ter os dados todos
classificados, faltam 10 exames. Mais significativo do que isso, esse
é um trabalho constante, a cada novo exame, as questões precisarão ser
classificadas, por isso, é importante automatizar, ao menos parte, do
processo.

Observações sobre a classificação estão no
\href{https://github.com/own-pt/oab-exams/blob/master/official/raw/README.org}{
  README} desse diretório.

\section{Tipos}

É possível modelar casos típicos de classificação.

\subsection{O enunciado dá a resposta}

Esse é o melhor tipo de questão. O próprio enunciado define com base
em qual legislação a questão deve ser respondida. Sabendo a
legislação, necessariamente a área está definida. Um exemplo disso é:

\begin{verbatim}
...
AREA ETHICS

Alexandre, advogado que exerce a profissão há muitos anos, é 
conhecido por suas atitudes corajosas, sendo respeitado pelos 
seus clientes e pelas autoridades com quem se relaciona por 
questões profissionais. Comentando sua atuação profissional, 
ele foi inquirido, por um dos seus filhos, se não deveria 
recusar a defesa de um indivíduo considerado impopular, bem 
como se não deveria ser mais obediente às autoridades, 
diante da possibilidade de retaliação. 
 
Sobre o caso apresentado, observadas as regras do Estatuto 
da OAB, assinale a opção correta indicada ao filho do 
advogado citado. 
 
OPTIONS
...
\end{verbatim}

Veja que o enunciado acima dá a resposta com \textit{observadas as
  regras do Estatuto da OAB}. Em outros casos, a menção à norma
aplicável é mais sútil. Um bom exemplo disso é:

\begin{verbatim}
...
AREA CONSTITUTIONAL

A Constituição declara que todos podem reunir-se em local 
aberto ao público. Algumas condições para que as reuniões se 
realizem são apresentadas nas alternativas a seguir, à exceção 
de uma. Assinale-a. 

OPTIONS
...
\end{verbatim}

Perceba que a Constituição é citada e isso contextualiza o que está
sendo perguntado. Mais sútil do que dizer \textit{com base na norma
  X}.

\subsection{Mapa de conceitos}\label{sec:mapa}

Ao longo do curso de Direito, os alunos são apresentados a vários
conceitos. Esses conceitos, normalmente, estão inseridos em uma
área. As vezes, um mesmo conceito pertence a duas áreas. Na hora da
classificação manual, muitas vezes, se aparecia determinado conceito,
isso era suficiente para definir a classificação da questão.

Um exemplo em Direito Empresarial é o conceito de \textbf{recuperação
  judicial}, muito frequente nas provas da OAB. Veja a questão abaixo:

\begin{verbatim}
...
AREA BUSINESS

Com relação às atribuições do Comitê de Credores, quando 
constituído no âmbito da recuperação judicial, assinale a 
afirmativa correta. 
 
OPTIONS
...
\end{verbatim}

Existe uma lei que disciplina a recuperação judicial (uma alternativa
ao processo de falência), lei 11.101 de 2005. A lei não foi citada,
mas o conceito foi e esse conceito é sacramentado como parte da área
de Direito Empresarial.

Um outro é exemplo é Direito Tributário, é comum que determinado
tributo seja citado no enunciado. Isso significa, praticamente, uma
classificação automática da área como TAXES.

\begin{verbatim}
...
AREA TAXES
 
Três irmãos são donos de um imóvel, em proporções iguais. 
Em relação ao IPTU, cada irmão 
 
OPTIONS
...
\end{verbatim}

\subsection{Nome de autores}

A abordagem é um pouco semelhante com o mapa de conceitos. Muitas
vezes, se determinado nome de um autor aparece é porque com certeza
determinada área está sendo citada. Isso é muito forte nas questões de
Filosofia, que foram introduzidas no exame recentemente, no ano de
2012.

Essa questão, por exemplo, cita o filósofo Kant. É muito improvável
que esse autor seja citado em qualquer outra área. Pode acontecer,
talvez, um gancho com Direito Constitucional, com Direitos Humanos,
com Direito Internacional e, talvez, até Direito Penal. Mas é muito
improvável considerando o histórico da prova. Normalmente, são poucas
questões de filosofia, tradicionalmente, são duas de 80.

\begin{verbatim}
...
AREA PHILOSOPHY

“Manter os próprios compromissos não constitui dever de 
virtude, mas dever de direito, a cujo cumprimento pode-se ser 
forçado. Mas prossegue sendo uma ação virtuosa (uma 
demonstração de virtude) fazê-lo mesmo quando nenhuma 
coerção possa ser aplicada. A doutrina do direito e a doutrina 
da virtude não são, consequentemente, distinguidas tanto por 
seus diferentes deveres, como pela diferença em sua 
legislação, a qual relaciona um motivo ou outro com a lei”. 
 
Pelo trecho acima podemos inferir que Kant estabelece uma 
relação entre o direito e a moral. A esse respeito, assinale a 
afirmativa correta.  
 
OPTIONS
...
\end{verbatim}

No entanto, nem sempre um autor é citado numa questão de
filosofia. Talvez isso torne questões dessa área uma das mais
difíceis! Talvez tratar como algo residual, como "else", se não for
nenhuma outra aŕea, bem provável que seja filosofia.

\begin{verbatim}
...
AREA PHILOSOPHY
 
Boa parte da doutrina jusfilosófica contemporânea associa a 
ideia de Direito ao conceito de razão prática ou sabedoria 
prática.  
 
Assinale a alternativa que apresenta o conceito correto de 
razão prática. 
 
OPTIONS

A) Uma forma de conhecimento científico (episteme) capaz 
de distinguir entre o verdadeiro e o falso. 

B) Uma técnica (techne) capaz de produzir resultados 
universalmente corretos e desejados. 

C) A manifestação de uma opinião (doxa) qualificada ou 
ponto de vista específico de um agente diante de um tema 
específico. 

D:CORRECT) A capacidade de bem deliberar (phronesis) a respeito de 
bens ou questões humanas. 
\end{verbatim}


\subsection{Número da questão na prova}

O número da questão ajuda muito na classificação. Tradicionalmente, a
prova tem um padrão na ordem em que as questões são apresentadas. Por
exemplo, nos últimos exames, a área de ética, normalmente, é a
primeira e a de Direito Processual Trabalhista, a última. Isso não
ocorre apenas para a primeira e última, mas por toda a prova!

Veja a questão abaixo. Ela trata sobre desapropriação no contexto de
reforma agrária. Esse é um assunto discutido em Direito Civil, Direito
Administrativo e Direito Constitucional.

Na hora da classificação, definitivamente, foi levado em conta que a
questão anterior era sobre constitucional e a que seguinte também era
sobre constiucional. Além disso, a questão era de número 14 e,
tradicionalmente, as questões de 13 a 20 são de Constitucional.

Mais do que isso, tecnicamente, o artigo 184 da Constituição em seu
\textit{caput} e no seu parágrafo primeiro respondem a questão:

\begin{verbatim}
Art. 184. Compete à União desapropriar por
interesse social, para fins de reforma agrária, o
imóvel rural que não esteja cumprindo sua função
social, mediante prévia e justa indenização em
títulos da dívida agrária, com cláusula de
preservação do valor real, resgatáveis no prazo
de até vinte anos, a partir do segundo ano de sua
emissão, e cuja utilização será definida em
lei.
§ 1º As benfeitorias úteis e necessárias serão
indenizadas em dinheiro.
\end{verbatim}


\begin{verbatim}
...
AREA CONSTITUTIONAL
  
Assinale a alternativa que completa corretamente o 
fragmento a seguir. 

A desapropriação para fins de reforma agrária ocorre 
mediante prévia e justa indenização  

OPTIONS 

A) em dinheiro, incluindo-se as benfeitorias úteis e 
necessárias. 

B) em dinheiro, mas as benfeitorias não são passíveis de 
indenização. 

C) em títulos da dívida agrária, incluindo-se as benfeitorias 
úteis e necessárias. 

D:CORRECT) em títulos da dívida agrária, mas as benfeitorias úteis e 
necessárias serão indenizadas em dinheiro
\end{verbatim}


\subsection{Instituições citadas}

Esse tipo de questão é semelhante ao tipo da
seção~\ref{sec:mapa}. Diferentemente do nome do autor ser um
referencial para a classificação da área, a menção a determinada
instituição é suficiente para a classificação quase automática. No
exemplo abaixo, o Tribunal Superior do Trabalho (TST) é citado:

\begin{verbatim}
...
AREA LABOUR
 
Rodrigo foi admitido pela empresa Dona Confecções, a título 
de experiência, por 45 dias. No 35º dia após a admissão, 
Rodrigo foi vítima de um acidente do trabalho de média 
proporção, que o obrigou ao afastamento por 18 dias. De 
acordo com o entendimento do TST: 
 
OPTIONS

A:CORRECT) Rodrigo não poderá ser dispensado pois, em razão do 
acidente do trabalho, possui garantia no emprego, mesmo 
no caso de contrato a termo.  

B) O contrato poderá ser rompido porque foi realizado por 
prazo determinado, de forma que nenhum fator, por mais 
relevante que seja, poderá elastecê-lo. 

C) Rodrigo poderá ser desligado porque a natureza jurídica da 
ruptura não será resilição unilateral, mas caducidade 
contratual, que é outra modalidade de rompimento. 

D) Rodrigo não pode ter o contrato rompido no termo final, 
pois em razão do acidente do trabalho sofrido, terá 
garantia no emprego até 5 meses após o retorno, 
conforme Lei previdenciária. 
\end{verbatim}

\section{Hard Cases}

Algumas questões são casos difíceis e servem como contra-exemplos ou
exceções para a modelagem feito acima.

Em alguns casos, será preciso combinar estratégias de classificação
citadas acima. Uma boa questão para ilustrar isso é:

\begin{verbatim}
...
AREA ADMINISTRATIVE

Paulo é servidor concursado da Câmara de Vereadores do 
município Beta há mais de quinze anos. Durante esse tempo, 
Paulo concluiu cursos de aperfeiçoamento profissional, 
graduou-se no curso de economia, exerceu cargos em 
comissão e foi promovido por merecimento. Todos esses 
fatores contribuíram para majorar sua remuneração. 
 
Considerando a disciplina constitucional a respeito dos 
servidores públicos, assinale a afirmativa correta. 
 
OPTIONS

A:CORRECT) O teto remuneratório aplicável a Paulo, servidor público 
municipal, corresponde ao subsídio do prefeito do 
município Beta. 

B) O teto remuneratório aplicável a Paulo, servidor público 
municipal, corresponde ao subsídio pago aos vereadores 
de Beta. 

C) Os acréscimos de caráter remuneratório, pagos a Paulo, 
como a gratificação por tempo de serviço e a gratificação 
adicional de qualificação profissional, não se submetem ao 
teto remuneratório. 

D) O teto remuneratório aplicável a Paulo não está sujeito a 
qualquer limitação, tendo em vista a necessidade de 
edição de lei complementar para a instituição do teto 
previsto na CRFB/88. 
\end{verbatim}

A resposta dessa questão está na Constituição, no artigo 37, inciso XI:

\begin{verbatim}
a remuneração e o subsídio dos ocupantes de
cargos, funções e empregos públicos da
administração direta, autárquica e fundacional,
dos membros de qualquer dos Poderes da União, dos
Estados, do Distrito Federal e dos Municípios, dos
detentores de mandato eletivo e dos demais
agentes políticos e os proventos, pensões ou
outra espécie remuneratória, percebidos
cumulativamente ou não, incluídas as vantagens 
pessoais ou de qualquer outra natureza, não 
poderão exceder o subsídio mensal, em espécie, 
dos Ministros do Supremo Tribunal Federal, 
aplicando-se como limite, nos Municípios, o 
subsídio do Prefeito (…)
\end{verbatim} 

No entanto, a classificação dessa questão, imho, é de Direito
Administrativo. Basicamente, porque esse artigo da Constituição, o 37,
é um dos grandes pilares do Direito Administrativo brasileiro. Essas
duas áreas, Direito Administrativo e Direito Constitucional tem grande
sinergia e forte interseção. Muitas vezes, o Direito Administrativo é
responsável por detalhar e especificar diretrizes passadas no texto
Constitucional.

Além disso, a numeração da questão também interfere. Ela está fora do
bloco de questões constitucionais. Essa é a questão 30 do XX Exame
Unificado (2016).

A questão anterior, questão 29, é claramente de Direito
Administrativo, pois trata de serviços públicos, impugnação de edital
e concessões:

\begin{verbatim}
ENUM Questão 29 
 
Determinada empresa apresenta impugnação ao edital de 
concessão do serviço público metroviário em determinado 
Estado, sob a alegação de que a estipulação do retorno ao 
poder concedente de todos os bens reversíveis já 
amortizados, quando do advento do termo final do contrato, 
ensejaria enriquecimento sem causa do Estado. 
 
Assinale a opção que indica o princípio que justifica tal 
previsão editalícia. 
 
OPTIONS

A) Desconcentração. 

B) Imperatividade. 

C:CORRECT) Continuidade dos Serviços Públicos. 

D) Subsidiariedade. 
\end{verbatim} 

A questão seguinte à questão 30, é a de número 31, e também é de
Direito Administrativo, pois trata de improbidade administrativa, tema
clássico de administrativo:

\begin{verbatim}
---
ENUM Questão 31 
 
O diretor-presidente de uma construtora foi procurado pelo 
gerente de licitações de uma empresa pública federal, que 
propôs a contratação direta de sua empresa, com dispensa de 
licitação, mediante o pagamento de uma “contribuição” de 2% 
(dois por cento) do valor do contrato, a ser depositado em 
uma conta no exterior. Contudo, após consumado o acerto, foi 
ele descoberto e publicado em revista de grande circulação. 
 
A respeito do caso descrito, assinale a afirmativa correta. 

OPTIONS

A) Somente o gerente de licitações da empresa pública, 
agente público, está sujeito a eventual ação de 
improbidade administrativa. 

B) Nem o diretor-presidente da construtora e nem o gerente 
de licitações da empresa pública, que não são agentes 
públicos, estão sujeitos a eventual ação de improbidade 
administrativa. 

C) O diretor-presidente da construtora, beneficiário do 
esquema, está sujeito a eventual ação de improbidade, 
mas o gerente da empresa pública, por não ser servidor 
público, não está sujeito a tal ação. 

D:CORRECT) O diretor-presidente da construtora e o gerente de 
licitações da empresa pública estão sujeitos a eventual 
ação de improbidade administrativa. 
\end{verbatim}

Portanto, pesou para a classificação de Direito Administrativo o
contexto na prova e saber da forte interseção de Direito
Administrativo e Constitucional. Como segunda opinião, veja esse blog
muitíssimo conhecido no mundo dos preparatórios para a
\href{https://www.estrategiaconcursos.com.br/blog/comentarios-administrativo-oab/}{oab}.

Essa questão foi analisada por um professor de Direito Administrativo
e a página em que a análise está inserida é chamada de
\textbf{Comentários à Prova do XX Exame de Ordem – Direito
  Administrativo}. A classificação desses especialistas corrobora a
que foi apresentada aqui.


\end{document}
