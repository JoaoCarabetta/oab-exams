\documentclass[12pt]{article}

\usepackage[utf8x]{inputenc} 
\usepackage[T1]{fontenc}
\usepackage{textcomp}
\usepackage{gensymb}
\usepackage{titling}
\usepackage{lipsum}
\usepackage{url}
\usepackage{graphicx}
\usepackage{listings}
\usepackage{color}
\usepackage[usenames,dvipsnames,svgnames,table]{xcolor}


\title{Fundamento questões}

\author{Pedro Delfino}


\begin{document}

\maketitle

\section*{Primeira leva de questões analisadas - 09/05/2017} 

\section{Questão 6 do XX° Exame Unificado (2016)} 

\subsection{Enunciado} 

Charles é presidente de certo Conselho Seccional da OAB. Não
obstante, no curso do mandato, Charles vê-se envolvido em
dificuldades no seu casamento com Emma, e decide renunciar
ao mandato, para dedicar-se às suas questões pessoais.


Sobre o caso, assinale a afirmativa correta.


OPTIONS


A) O sucessor de Charles deverá ser eleito pelo Conselho
Federal da OAB, dentre os membros do Conselho Seccional
respectivo.


\textbf{B:CORRECT) O sucessor de Charles deverá ser eleito pelo Conselho Seccional respectivo, dentre seus membros.}

C) O sucessor de Charles deverá ser eleito pela Subseção
respectiva, dentre seus membros.

D) O sucessor de Charles deverá ser eleito por votação direta
dos advogados regularmente inscritos perante o Conselho
Seccional respectivo.


\subsection{Fundamento}

O fundamento dessa questão está na Lei 8906, o estatuto da OAB. Como dito no e-mail, grande parte das questões da seção de Ética são respondidas com essa lei. O fundamento está no parágrafo único do artigo 66:

Art. 66. Extingue-se o mandato automaticamente, antes do seu término, quando:

I - ocorrer qualquer hipótese de cancelamento de inscrição ou de licenciamento do profissional;

II - o titular sofrer condenação disciplinar;

III - o titular faltar, sem motivo justificado, a três reuniões ordinárias consecutivas de cada órgão deliberativo do conselho ou da diretoria da Subseção ou da Caixa de Assistência dos Advogados, não podendo ser reconduzido no mesmo período de mandato.

\textbf{Parágrafo único. Extinto qualquer mandato, nas hipóteses deste artigo, cabe ao Conselho Seccional escolher o substituto, caso não haja suplente.
}

\section{Questão 4 do XIX° Exame Unificado da Ordem (2016)}


\subsection{Enunciado}
Formaram-se em uma Faculdade de Direito, na mesma turma, 
Luana, Leonardo e Bruno. Luana, 35 anos, já exercia função de 
gerência em um banco quando se graduou. Leonardo, 30 
anos, é prefeito do município de Pontal. Bruno, 28 anos, é 
policial militar no mesmo município. Os três pretendem 
praticar atividades privativas de advocacia.  
 
Considerando as incompatibilidades e impedimentos ao 
exercício da advocacia, assinale a opção correta. 
 
OPTIONS

A) Luana não está proibida de exercer a advocacia, pois é 
empregada de instituição privada, inexistindo 
impedimentos ou incompatibilidades. 

B) Bruno, como os servidores públicos, apenas é impedido de 
exercer a advocacia contra a Fazenda Pública que o 
remunera. 

\textbf{C:CORRECT) Os três graduados, Luana, Leonardo e Bruno, exercem 
funções incompatíveis com a advocacia, sendo 
determinada a proibição total de exercício das atividades 
privativas de advogado. }

D) Leonardo é impedido de exercer a advocacia apenas 
contra ou em favor de pessoas jurídicas de direito público, 
empresas públicas, sociedades de economia mista, 
fundações públicas, entidades paraestatais ou empresas 
concessionárias ou permissionárias de serviço público. 

\subsection{Fundamento}

A resposta para essa questão está no artigo 28 da Lei 8906. Essa lei tem vários incisos, cada uma das pessoas citadas cai em uma hipótese. Vou destacar em negrito:

Art. 28. A advocacia é incompatível, mesmo em causa própria, com as seguintes atividades:

\textbf{I - chefe do Poder Executivo e membros da Mesa do Poder Legislativo e seus substitutos legais;
}
II - membros de órgãos do Poder Judiciário, do Ministério Público, dos tribunais e conselhos de contas, dos juizados especiais, da justiça de paz, juízes classistas, bem como de todos os que exerçam função de julgamento em órgãos de deliberação coletiva da administração pública direta e indireta;  (Vide ADIN 1127-8)

III - ocupantes de cargos ou funções de direção em Órgãos da Administração Pública direta ou indireta, em suas fundações e em suas empresas controladas ou concessionárias de serviço público;

IV - ocupantes de cargos ou funções vinculados direta ou indiretamente a qualquer órgão do Poder Judiciário e os que exercem serviços notariais e de registro;

\textbf{V - ocupantes de cargos ou funções vinculados direta ou indiretamente a atividade policial de qualquer natureza;
}
VI - militares de qualquer natureza, na ativa;

VII - ocupantes de cargos ou funções que tenham competência de lançamento, arrecadação ou fiscalização de tributos e contribuições parafiscais;

\textbf{VIII - ocupantes de funções de direção e gerência em instituições financeiras, inclusive privadas.
}
§ 1º A incompatibilidade permanece mesmo que o ocupante do cargo ou função deixe de exercê-lo temporariamente.

§ 2º Não se incluem nas hipóteses do inciso III os que não detenham poder de decisão relevante sobre interesses de terceiro, a juízo do conselho competente da OAB, bem como a administração acadêmica diretamente relacionada ao magistério jurídico.


\section{Questão 1 do XVIII° Exame Unificado da Ordem (2015) } 

\subsection{Enunciado} 

Paulo é contratado por Pedro para promover ação com pedido 
condenatório em face de Alexandre, por danos causados ao 
animal de sua propriedade. Em decorrência do processo, 
houve condenação do réu ao pagamento de indenização ao 
autor, fixados honorários de sucumbência correspondentes a 
dez por cento do apurado em cumprimento de sentença. O 
réu ofertou apelação contra a sentença proferida na fase 
cognitiva. Ainda pendente o julgamento do recurso, Pedro 
decide revogar o mandato judicial conferido a Paulo, 
desobrigando-se de pagar os honorários contratualmente 
ajustados. 

Nos termos do Código de Ética da OAB, a revogação do 
mandato judicial, por vontade de Pedro, 
 
OPTIONS

\textbf{A:CORRECT) não o desobriga do pagamento das verbas honorárias 
contratadas. }

B) desobriga-o do pagamento das verbas honorárias 
contratadas. 

C) desobriga-o do pagamento das verbas honorárias 
contratadas e da verba sucumbencial. 

D) não o desobriga do pagamento das verbas honorárias 
sucumbenciais, mas o desobriga das verbas contratadas. 

\subsection{Fundamento} 

Essa é uma questão que não está fundamentada na Lei 8906. Diferentemente das anteriores, o fundamento da resposta desta pergunta está no Artigo 14 do Código de Ética e disciplina da OAB:

Art. 14. A revogação do mandato judicial por vontade do cliente não o desobriga do pagamento  das  verbas  honorárias  contratadas,  bem  como  não  retira  o  direito  do advogado  de  receber  o  quanto  lhe  seja  devido  em eventual  verba  honorária  de sucumbência, calculada proporcionalmente, em face do serviço efetivamente prestado. 

\subsection{Observação} 

A Lei 8906 versa sobre o Estatuto da Advocacia e sobre a instituição OAB, trata-se de uma norma aprovada no Congresso nacional. O Código de Ética e Disciplina da OAB tem status apenas infra-legal e foi instituído pela OAB. Da mesma forma, o Regulamento Geral da OAB também não é uma lei aprovada no Congresso e sim uma norma instituída pela própria OAB.

\section{Questão 2 do XVII° Exame Unificado da Ordem (2015)} 

\subsection{Enunciado} 

Os atos e contratos constitutivos de pessoas jurídicas, para sua 
admissão em registro, em não se tratando de empresas de 
pequeno porte e de microempresas, consoante o Estatuto da 
Advocacia, devem 
 
OPTIONS

A) apresentar os dados do contador responsável. 

B) permitir a participação de outros profissionais liberais. 

\textbf{C:CORRECT) conter o visto do advogado.}

D) indicar o advogado que representará a sociedade.

\subsection{Fundamento}


Lei 8906:

Art. 1º São atividades privativas de advocacia:

I - a postulação a qualquer órgão do Poder Judiciário e aos juizados especiais;        (Vide ADIN 1.127-8)

II - as atividades de consultoria, assessoria e direção jurídicas.

§ 1º Não se inclui na atividade privativa de advocacia a impetração de habeas corpus em qualquer instância ou tribunal.

\textbf{§ 2º Os atos e contratos constitutivos de pessoas jurídicas, sob pena de nulidade, só podem ser admitidos a registro, nos órgãos competentes, quando visados por advogados.}

   § 3º É vedada a divulgação de advocacia em conjunto com outra atividade.


\section{Questão 1 do XVI° Exame Unificado da Ordem (2015)} 

\subsection{Enunciado} 

Bernardo é bacharel em Direito, mas não está inscrito nos 
quadros da Ordem dos Advogados do Brasil, apesar de 
aprovado no Exame de Ordem. Não obstante, tem atuação na 
área de advocacia, realizando consultorias e assessorias 
jurídicas. 
 
A partir da hipótese apresentada, nos termos do Regulamento 
Geral da Ordem dos Advogados do Brasil, assinale a afirmativa 
correta. 
 
OPTIONS

A) Tal conduta é permitida, por ter o bacharel logrado 
aprovação no Exame de Ordem. 

B) Tal conduta é proibida, por ser equiparada à captação de 
clientela. 

C) Tal conduta é permitida mediante autorização do 
Presidente da Seccional da Ordem dos Advogados do 
Brasil. 

\textbf{D:CORRECT) Tal conduta é proibida, tendo em vista a ausência de inscrição na Ordem dos Advogados do Brasil.}

\subsection{Fundamento}

A base dessa resposta está no artigo 3° da lei 8906, \textit{caput}:

\textbf{Art. 3º O exercício da atividade de advocacia no território brasileiro e a denominação de advogado são privativos dos inscritos na Ordem dos Advogados do Brasil (OAB),}

§ 1º Exercem atividade de advocacia, sujeitando-se ao regime desta lei, além do regime próprio a que se subordinem, os integrantes da Advocacia-Geral da União, da Procuradoria da Fazenda Nacional, da Defensoria Pública e das Procuradorias e Consultorias Jurídicas dos Estados, do Distrito Federal, dos Municípios e das respectivas entidades de administração indireta e fundacional.

§ 2º O estagiário de advocacia, regularmente inscrito, pode praticar os atos previstos no art. 1º, na forma do regimento geral, em conjunto com advogado e sob responsabilidade deste.


\section{Questão 4 do XV° Exame Unificado da Ordem (2014)} 

\subsection{Enunciado} 

Sobre a prescrição da pretensão punitiva das infrações 
disciplinares, assinale a afirmativa correta.
 
OPTIONS

\textbf{A:CORRECT) A pretensão punitiva quanto às infrações disciplinares
prescreve em cinco anos, contados da data da constatação 
oficial do fato, interrompendo
processo disciplinar ou pela notificação válida do 
representado.} 

B) A pretensão punitiva das infrações disciplinares prescreve 
em três anos, contados da data da constatação oficial do 
fato, interrompendo-se pela instauração de processo 
disciplinar ou pela notificação v

C) A pretensão punitiva das infrações disciplinares é 
imprescritível. 

D) A pretensão punitiva das infrações disciplinares prescreve 
em cinco anos, contados da data da constatação oficial do 
fato, não havendo previsão legal de marco interruptivo de
tal prazo prescricional. 

\subsection{Fundamento}

A resposta está no caput do artigo 43 da Lei 8906:

\textbf{Art. 43. A pretensão à punibilidade das infrações disciplinares prescreve em cinco anos, contados da data da constatação oficial do fato.}

§ 1º Aplica-se a prescrição a todo processo disciplinar paralisado por mais de três anos, pendente de despacho ou julgamento, devendo ser arquivado de ofício, ou a requerimento da parte interessada, sem prejuízo de serem apuradas as responsabilidades pela paralisação.

§ 2º A prescrição interrompe-se:

I - pela instauração de processo disciplinar ou pela notificação válida feita diretamente ao representado;

II - pela decisão condenatória recorrível de qualquer órgão julgador da OAB. 

\subsection{Observação}

Essa me parece uma boa questão para text-entailment.

\noindent\makebox[\linewidth]{\rule{\paperwidth}{0.4pt}}


\section*{Segunda leva de questões analisadas - 04/08/2017} 


\section{Questão 2 do XIV° Exame Unificado da Ordem (2014)}

\subsection{Enunciado}

Andrea e Luciano trocam missivas intermitentes, cujo 
conteúdo diz respeito a processo judicial em que a primeira é 
autora, e o segundo, seu advogado. A parte contrária, ciente 
da troca de informações entre eles, requer ao Juízo que esses 
documentos sejam anexados aos autos do processo em que 
litigam. 
 
Sob a perspectiva do Código de Ética e Disciplina da 
Advocacia, as comunicações epistolares trocadas entre 
advogado e cliente 

OPTIONS

A) constituem documentos públicos a servirem como prova 
em Juízo. 

\textbf{B:CORRECT) são presumidas confidenciais, não podendo ser reveladas a terceiros. }

C) podem ser publicizadas, de acordo com a prudência do 
advogado. 

D) devem ser mantidas em sigilo até o perecimento do 
advogado.

\subsection{Fundamento}

Está no código de ética da OAB, no parágrafo único do Artigo 27 (não está naquela Lei 8906):

Art. 27. As  confidências  feitas  ao  advogado  pelo  cliente  podem  ser  utilizadas  nos limites da necessidade da defesa, desde que autorizado aquele pelo constituinte.  

\textbf{Parágrafo   único.   Presumem-se   confidenciais   as   comunicações   epistolares   entre advogado e cliente, as quais não podem ser reveladas a terceiros.}



\section{Questão 1 do XIII° Exame Unificado da Ordem (2014)}

\subsection{Enunciado}

O advogado Carlos pretende substabelecer os poderes que lhe 
foram conferidos pelo seu cliente Eduardo, sem reserva de 
poderes, pois pretende realizar uma longa viagem, sem saber 
a data do retorno, não pretendendo manter compromissos 
profissionais. 
 
Nos termos das normas do Código de Ética, tal ato deve 
 
OPTIONS

A) prescindir do conhecimento do cliente por ser ato 
privativo. 

\textbf{B:CORRECT) ser comunicado ao cliente de modo inequívoco.}

C) ser realizado por tempo determinado. 

D) implicar na devolução dos honorários pagos 
antecipadamente pelo cliente.

\subsection{Fundamento}

Está no Código de Ética da OAB (não está naquela Lei 8906), artigo 24 e parágrafo único. Bem direto como vc pode ver:

Art. 24. O substabelecimento  do  mandato,  com  reserva  de poderes,  é  ato  pessoal  do advogado da causa.  

\textbf{§ 1º O  substabelecimento  do  mandato  sem  reservas  de  poderes  exige  o  prévio  e inequívoco conhecimento do cliente.}

§ 2º O substabelecido com reserva de poderes deve ajustar antecipadamente seus honorários com o substabelecente. 
 
\section{Questão 1 do XII° Exame Unificado da Ordem (2013)}

\subsection{Enunciado}

O advogado João foi contratado por José para atuar em 
determinada ação indenizatória. Ao ter vista dos autos em 
cartório, percebeu que José já estava representado por outro 
advogado na causa. Mesmo assim, considerando que já havia 
celebrado contrato com José, mas sem contatar o advogado 
que se encontrava até então constituído, apresentou petição 
requerendo juntada da procuração pela qual José lhe 
outorgara poderes para atuar na causa, bem como a retirada 
dos autos em carga, para que pudesse examiná-los com 
profundidade em seu escritório.  
 
Com base no caso apresentado, assinale a afirmativa correta. 
 
OPTIONS

A) O advogado João não cometeu infração disciplinar, pois 
apenas requereu a juntada de procuração e realizou carga 
dos autos do processo, sem apresentar petição com 
conteúdo relevante para o deslinde da controvérsia. 

B) O advogado João cometeu infração disciplinar, não por ter 
requerido a juntada de procuração nos autos, mas sim por 
ter realizado carga dos autos do processo em que já havia 
advogado constituído. 

C) O advogado João não cometeu infração disciplinar, pois, ao 
requerer a juntada da procuração nos autos, já havia 
celebrado contrato com José. 

\textbf{D:CORRECT) O advogado João cometeu infração disciplinar prevista no Código de Ética e Disciplina da OAB, pois não pode aceitar 
procuração de quem já tenha patrono constituído, sem 
prévio conhecimento do mesmo.}

\subsection{Fundamento}

Está no Código de Ética da OAB (não está naquela Lei 8906). Bem direto como vc pode ver:

Art. 11. O   advogado   não   deve   aceitar   procuração   de   quem   já   tenha   patrono constituído, sem prévio conhecimento deste, salvo por motivo justo ou para adoção de medidas judiciais urgentes e inadiáveis.  

\section{Questão 1 do XI° Exame Unificado da Ordem (2013)}

\subsection{Enunciado}

Christiana, advogada recém-formada, está em dúvida quanto 
ao seu futuro profissional, porque, embora possua habilidade 
para a advocacia privada, teme a natural instabilidade da 
profissão. Por força dessas circunstâncias, pretende obter um 
emprego ou cargo público que lhe permita o exercício 
concomitante da profissão que abraçou. Por força disso, 
necessita, diante dos requisitos usualmente exigidos, 
comprovar sua efetiva atividade na advocacia.  
 
Diante desse contexto, de acordo com as normas do 
Regulamento Geral do Estatuto da Advocacia e da OAB, 
assinale a afirmativa correta. 
 
OPTIONS

A) O efetivo exercício da advocacia comprova-se pela atuação 
em um processo por ano, desde que o advogado subscreva 
uma peça privativa de advogado. 

\textbf{B:CORRECT) O efetivo exercício da advocacia exige a atuação anual mínima em cinco causas distintas, que devem ser 
comprovadas por cópia autenticada de atos privativos. }

C) A atividade efetiva da advocacia, como representante 
judicial ou extrajudicial, cinge-se a dois atos por ano. 

D) O advogado deve comprovar, anualmente, a atuação em 
atos privativos, mediante declaração do Juiz onde atue, de 
três atos judiciais.

\subsection{Fundamento}

Está no código de ética da OAB (não está naquela Lei 8906). Bem direto como vc pode ver:

\textbf{Art. 5º Considera-se efetivo exercício da atividade de advocacia a participação anual mínima em cinco atos privativos previstos no artigo 1º do Estatuto, em causas ouquestões distintas.}

Parágrafo único. A comprovação do efetivo exercício faz-se mediante:

a) certidão expedida por cartórios ou secretarias judiciais;

b) cópia autenticada de atos privativos;

c) certidão expedida pelo órgão público no qual o advogado exerça função privativa do seu ofício, indicando os atos praticados.

\section{Questão 9 do XI° Exame Unificado da Ordem (2013)}

\subsection{Enunciado}

Úrsula, advogada com larga experiência profissional, necessita 
atualizar o seu arquivo de causas. Assim, requer o 
desarquivamento de determinados autos processuais de 
processo findo de um cliente, que tramitou sob sigilo, mas de 
época anterior à sua atuação. Ao dirigir-se ao cartório judicial, 
é surpreendida pela exigência de procuração com poderes 
especiais para retirar os autos.  

Nos termos do Estatuto da Advocacia, é direito do advogado 
retirar autos de processos findos 
 
OPTIONS

A) com procuração, inseridos poderes gerais, pelo prazo de 
cinco dias.  

B) com procuração, com poderes especiais, pelo prazo de 
quinze dias. 

C) sem procuração, com autorização do escrivão do cartório, 
pelo prazo de dez dias. 

\textbf{D:CORRECT) sem procuração, pelo prazo de dez dias.}

\subsection{Fundamento}

Está naquela Lei 8906. O artigo 7 tem diversos incisos. A resposta está no inciso XVI, destacado em azul. Bem direto como vc pode ver:

Art. 7º São direitos do advogado:

I - exercer, com liberdade, a profissão em todo o território nacional;

II - ter respeitada, em nome da liberdade de defesa e do sigilo profissional, a inviolabilidade de seu escritório ou local de trabalho, de seus arquivos e dados, de sua correspondência e de suas comunicações, inclusive telefônicas ou afins, salvo caso de busca ou apreensão determinada por magistrado e acompanhada de representante da OAB;

II – a inviolabilidade de seu escritório ou local de trabalho, bem como de seus instrumentos de trabalho, de sua correspondência escrita, eletrônica, telefônica e telemática, desde que relativas ao exercício da advocacia; (Redação dada pela Lei nº 11.767, de 2008)

III - comunicar-se com seus clientes, pessoal e reservadamente, mesmo sem procuração, quando estes se acharem presos, detidos ou recolhidos em estabelecimentos civis ou militares, ainda que considerados incomunicáveis;

IV - ter a presença de representante da OAB, quando preso em flagrante, por motivo ligado ao exercício da advocacia, para lavratura do auto respectivo, sob pena de nulidade e, nos demais casos, a comunicação expressa à seccional da OAB;

V - não ser recolhido preso, antes de sentença transitada em julgado, senão em sala de Estado Maior, com instalações e comodidades condignas, assim reconhecidas pela OAB, e, na sua falta, em prisão domiciliar;         (Vide ADIN 1.127-8)

VI - ingressar livremente:

a) nas salas de sessões dos tribunais, mesmo além dos cancelos que separam a parte reservada aos magistrados;

b) nas salas e dependências de audiências, secretarias, cartórios, ofícios de justiça, serviços notariais e de registro, e, no caso de delegacias e prisões, mesmo fora da hora de expediente e independentemente da presença de seus titulares;

c) em qualquer edifício ou recinto em que funcione repartição judicial ou outro serviço público onde o advogado deva praticar ato ou colher prova ou informação útil ao exercício da atividade profissional, dentro do expediente ou fora dele, e ser atendido, desde que se ache presente qualquer servidor ou empregado;

d) em qualquer assembléia ou reunião de que participe ou possa participar o seu cliente, ou perante a qual este deva comparecer, desde que munido de poderes especiais;

VII - permanecer sentado ou em pé e retirar-se de quaisquer locais indicados no inciso anterior, independentemente de licença;

VIII - dirigir-se diretamente aos magistrados nas salas e gabinetes de trabalho, independentemente de horário previamente marcado ou outra condição, observando-se a ordem de chegada;

IX - sustentar oralmente as razões de qualquer recurso ou processo, nas sessões de julgamento, após o voto do relator, em instância judicial ou administrativa, pelo prazo de quinze minutos, salvo se prazo maior for concedido;       (Vide ADIN 1.127-8)      (Vide ADIN 1.105-7)

X - usar da palavra, pela ordem, em qualquer juízo ou tribunal, mediante intervenção sumária, para esclarecer equívoco ou dúvida surgida em relação a fatos, documentos ou afirmações que influam no julgamento, bem como para replicar acusação ou censura que lhe forem feitas;

XI - reclamar, verbalmente ou por escrito, perante qualquer juízo, tribunal ou autoridade, contra a inobservância de preceito de lei, regulamento ou regimento;

XII - falar, sentado ou em pé, em juízo, tribunal ou órgão de deliberação coletiva da Administração Pública ou do Poder Legislativo;

XIII - examinar, em qualquer órgão dos Poderes Judiciário e Legislativo, ou da Administração Pública em geral, autos de processos findos ou em andamento, mesmo sem procuração, quando não estejam sujeitos a sigilo, assegurada a obtenção de cópias, podendo tomar apontamentos;

XIV - examinar em qualquer repartição policial, mesmo sem procuração, autos de flagrante e de inquérito, findos ou em andamento, ainda que conclusos à autoridade, podendo copiar peças e tomar apontamentos;

XIV - examinar, em qualquer instituição responsável por conduzir investigação, mesmo sem procuração, autos de flagrante e de investigações de qualquer natureza, findos ou em andamento, ainda que conclusos à autoridade, podendo copiar peças e tomar apontamentos, em meio físico ou digital;         (Redação dada pela Lei nº 13.245, de 2016)

XV - ter vista dos processos judiciais ou administrativos de qualquer natureza, em cartório ou na repartição competente, ou retirá-los pelos prazos legais;

\textbf{XVI - retirar autos de processos findos, mesmo sem procuração, pelo prazo de dez dias;}

XVII - ser publicamente desagravado, quando ofendido no exercício da profissão ou em razão dela;

XVIII - usar os símbolos privativos da profissão de advogado;

XIX - recusar-se a depor como testemunha em processo no qual funcionou ou deva funcionar, ou sobre fato relacionado com pessoa de quem seja ou foi advogado, mesmo quando autorizado ou solicitado pelo constituinte, bem como sobre fato que constitua sigilo profissional;

XX - retirar-se do recinto onde se encontre aguardando pregão para ato judicial, após trinta minutos do horário designado e ao qual ainda não tenha comparecido a autoridade que deva presidir a ele, mediante comunicação protocolizada em juízo.

XXI - assistir a seus clientes investigados durante a apuração de infrações, sob pena de nulidade absoluta do respectivo interrogatório ou depoimento e, subsequentemente, de todos os elementos investigatórios e probatórios dele decorrentes ou derivados, direta ou indiretamente, podendo, inclusive, no curso da respectiva apuração:         (Incluído pela Lei nº 13.245, de 2016)

a) apresentar razões e quesitos;         (Incluído pela Lei nº 13.245, de 2016)

b) (VETADO).         (Incluído pela Lei nº 13.245, de 2016)

§ 1º Não se aplica o disposto nos incisos XV e XVI:

1) aos processos sob regime de segredo de justiça;

2) quando existirem nos autos documentos originais de difícil restauração ou ocorrer circunstância relevante que justifique a permanência dos autos no cartório, secretaria ou repartição, reconhecida pela autoridade em despacho motivado, proferido de ofício, mediante representação ou a requerimento da parte interessada;

3) até o encerramento do processo, ao advogado que houver deixado de devolver os respectivos autos no prazo legal, e só o fizer depois de intimado.

§ 2º O advogado tem imunidade profissional, não constituindo injúria, difamação ou desacato puníveis qualquer manifestação de sua parte, no exercício de sua atividade, em juízo ou fora dele, sem prejuízo das sanções disciplinares perante a OAB, pelos excessos que cometer.        (Vide ADIN 1.127-8)

§ 3º O advogado somente poderá ser preso em flagrante, por motivo de exercício da profissão, em caso de crime inafiançável, observado o disposto no inciso IV deste artigo.

§ 4º O Poder Judiciário e o Poder Executivo devem instalar, em todos os juizados, fóruns, tribunais, delegacias de polícia e presídios, salas especiais permanentes para os advogados, com uso e controle assegurados à OAB.     (Vide ADIN 1.127-8)

§ 5º No caso de ofensa a inscrito na OAB, no exercício da profissão ou de cargo ou função de órgão da OAB, o conselho competente deve promover o desagravo público do ofendido, sem prejuízo da responsabilidade criminal em que incorrer o infrator.

§ 6o  Presentes indícios de autoria e materialidade da prática de crime por parte de advogado, a autoridade judiciária competente poderá decretar a quebra da inviolabilidade de que trata o inciso II do caput deste artigo, em decisão motivada, expedindo mandado de busca e apreensão, específico e pormenorizado, a ser cumprido na presença de representante da OAB, sendo, em qualquer hipótese, vedada a utilização dos documentos, das mídias e dos objetos pertencentes a clientes do advogado averiguado, bem como dos demais instrumentos de trabalho que contenham informações sobre clientes.       (Incluído pela Lei nº 11.767, de 2008)

§ 7o  A ressalva constante do § 6o deste artigo não se estende a clientes do advogado averiguado que estejam sendo formalmente investigados como seus partícipes ou co-autores pela prática do mesmo crime que deu causa à quebra da inviolabilidade.       (Incluído pela Lei nº 11.767, de 2008)

§ 8o  (VETADO)       (Incluído pela Lei nº 11.767, de 2008)

§ 9o  (VETADO)       (Incluído pela Lei nº 11.767, de 2008)

§ 10.  Nos autos sujeitos a sigilo, deve o advogado apresentar procuração para o exercício dos direitos de que trata o inciso XIV.          (Incluído pela Lei nº 13.245, de 2016)

§ 11.  No caso previsto no inciso XIV, a autoridade competente poderá delimitar o acesso do advogado aos elementos de prova relacionados a diligências em andamento e ainda não documentados nos autos, quando houver risco de comprometimento da eficiência, da eficácia ou da finalidade das diligências.          (Incluído pela Lei nº 13.245, de 2016)

§ 12.  A inobservância aos direitos estabelecidos no inciso XIV, o fornecimento incompleto de autos ou o fornecimento de autos em que houve a retirada de peças já incluídas no caderno investigativo implicará responsabilização criminal e funcional por abuso de autoridade do responsável que impedir o acesso do advogado com o intuito de prejudicar o exercício da defesa, sem prejuízo do direito subjetivo do advogado de requerer acesso aos autos ao juiz competente.         (Incluído pela Lei nº 13.245, de 2016)

\section{Questão 7 do XI° Exame Unificado da Ordem (2013)}

\subsection{Enunciado}

José é advogado de João em processo judicial que este 
promove contra Matheus. Encantado com as sucessivas 
campanhas de conciliação, busca obter o apoio do réu para 
um acordo, sem consultar previamente o patrono da parte 
contrária, Valter.  
 
Nos termos do Código de Ética, deve o advogado 
 
OPTIONS

A) buscar a conciliação a qualquer preço por ser um objetivo 
da moderna Jurisdição.  

\textbf{B:CORRECT) abster-se de entender-se diretamente com a parte adversa que tenha patrono constituído, sem o assentimento deste.} 

C) entender-se com as partes na presença de autoridade sem 
necessidade de comunicação ao ex adverso.  

D) participar de campanhas de conciliação e, caso infrutíferas, 
tentar o acordo extrajudicial diretamente com a parte 
contrária.

\subsection{Fundamento}

O próprio enunciado sugere que a resposta está no Código de Ética:

Art. 2° - O advogado, indispensavel a administracao da Justica, e defensor do estadodemocratico de direito, da cidadania, da moralidade publica, da Justica e da paz social, subordinando a atividade do seu Ministerio Privado a elevada funcao publica que exerce. Paragrafo unico. Sao deveres do advogado:

I - preservar, em sua conduta, a honra, a nobreza e a dignidade da profissao, zelando pelo seu carater de essencialidade e indispensabilidade;

II - atuar com destemor, independencia, honestidade, decoro, veracidade, lealdade, dignidade e boa-fe;

III - velar por sua reputação pessoal e profissional;

IV - empenhar-se, permanentemente, em seu aperfeiçoamento pessoal e profissional;

V - contribuir para o aprimoramento das instituições, do Direito e das leis;

VI - estimular a conciliação entre os litigantes, prevenindo, sempre que possível, a instauração de litígios;

VII - aconselhar o cliente a não ingressar em aventura judicial;

VIII - abster-se de:

a) utilizar de influência indevida, em seu benefício ou do cliente;

b) patrocinar interesses ligados a outras atividades estranhas à advocacia, em que também atue;

c) vincular o seu nome a empreendimentos de cunho manifestamente duvidoso;

d) emprestar concurso aos que atentem contra a ética, a moral, a honestidade e a dignidade da pessoa humana;

e) entender-se diretamente com a parte adversa que tenha patrono constituído, sem o assentimento deste.

IX - pugnar pela solução dos problemas da cidadania e pela efetivação dos seus direitos individuais, coletivos e difusos, no âmbito da comunidade

\subsection{Observação}

Essa questã indica onde a resposta está no próprio enunciado. Isso não é tão comum.

\section{Questão 5 do X° Exame Unificado da Ordem (2013)}

\subsection{Enunciado}

A advogada Maria solicitou, no cartório de determinada vara 
cível, ter vista e extrair cópias dos autos de processo não 
sujeito a sigilo. O serventuário a quem foi feita a solicitação 
afirmou que Maria não havia juntado procuração aos autos do 
processo em questão e, em razão disso, apenas poderia ter 
vista dos autos e que lhe seria vedada a extração de cópias. 
 
A partir do caso apresentado, assinale a afirmativa correta.  
 
OPTIONS

\textbf{A:CORRECT) O serventuário não agiu corretamente. Mesmo não 
estando constituída nos autos do processo, Maria pode ter 
vista e obter cópias dos autos do processo, já que o 
mesmo não está sujeito a sigilo.  }

B) O serventuário agiu corretamente. O advogado não 
constituído nos autos de determinado processo apenas 
pode ter vista dos mesmos em balcão, mas não pode 
retirá-los de cartório para extração de cópias. 

C) O serventuário não agiu corretamente. Tendo em vista que 
Maria não estava constituída nos autos e que não poderia 
retirá-los de cartório para a extração de cópias, o 
serventuário deveria ter providenciado pessoalmente as 
cópias de que Maria necessitava. 

D) O serventuário não agiu corretamente. Tendo em vista que 
Maria não estava constituída nos autos do processo, não 
poderia sequer ter vista dos mesmos.

\subsection{Fundamento}

Outra questão que vai nos direitos do advogado, a resposta está em um dos incisos do Art 7° da Lei 8906, no inciso 13:

Art. 7º São direitos do advogado:

I - exercer, com liberdade, a profissão em todo o território nacional;

II - ter respeitada, em nome da liberdade de defesa e do sigilo profissional, a inviolabilidade de seu escritório ou local de trabalho, de seus arquivos e dados, de sua correspondência e de suas comunicações, inclusive telefônicas ou afins, salvo caso de busca ou apreensão determinada por magistrado e acompanhada de representante da OAB;

II – a inviolabilidade de seu escritório ou local de trabalho, bem como de seus instrumentos de trabalho, de sua correspondência escrita, eletrônica, telefônica e telemática, desde que relativas ao exercício da advocacia; (Redação dada pela Lei nº 11.767, de 2008)

III - comunicar-se com seus clientes, pessoal e reservadamente, mesmo sem procuração, quando estes se acharem presos, detidos ou recolhidos em estabelecimentos civis ou militares, ainda que considerados incomunicáveis;

IV - ter a presença de representante da OAB, quando preso em flagrante, por motivo ligado ao exercício da advocacia, para lavratura do auto respectivo, sob pena de nulidade e, nos demais casos, a comunicação expressa à seccional da OAB;

V - não ser recolhido preso, antes de sentença transitada em julgado, senão em sala de Estado Maior, com instalações e comodidades condignas, assim reconhecidas pela OAB, e, na sua falta, em prisão domiciliar;         (Vide ADIN 1.127-8)

VI - ingressar livremente:

a) nas salas de sessões dos tribunais, mesmo além dos cancelos que separam a parte reservada aos magistrados;

b) nas salas e dependências de audiências, secretarias, cartórios, ofícios de justiça, serviços notariais e de registro, e, no caso de delegacias e prisões, mesmo fora da hora de expediente e independentemente da presença de seus titulares;

c) em qualquer edifício ou recinto em que funcione repartição judicial ou outro serviço público onde o advogado deva praticar ato ou colher prova ou informação útil ao exercício da atividade profissional, dentro do expediente ou fora dele, e ser atendido, desde que se ache presente qualquer servidor ou empregado;

d) em qualquer assembléia ou reunião de que participe ou possa participar o seu cliente, ou perante a qual este deva comparecer, desde que munido de poderes especiais;

VII - permanecer sentado ou em pé e retirar-se de quaisquer locais indicados no inciso anterior, independentemente de licença;

VIII - dirigir-se diretamente aos magistrados nas salas e gabinetes de trabalho, independentemente de horário previamente marcado ou outra condição, observando-se a ordem de chegada;

IX - sustentar oralmente as razões de qualquer recurso ou processo, nas sessões de julgamento, após o voto do relator, em instância judicial ou administrativa, pelo prazo de quinze minutos, salvo se prazo maior for concedido;       (Vide ADIN 1.127-8)      (Vide ADIN 1.105-7)

X - usar da palavra, pela ordem, em qualquer juízo ou tribunal, mediante intervenção sumária, para esclarecer equívoco ou dúvida surgida em relação a fatos, documentos ou afirmações que influam no julgamento, bem como para replicar acusação ou censura que lhe forem feitas;

XI - reclamar, verbalmente ou por escrito, perante qualquer juízo, tribunal ou autoridade, contra a inobservância de preceito de lei, regulamento ou regimento;

XII - falar, sentado ou em pé, em juízo, tribunal ou órgão de deliberação coletiva da Administração Pública ou do Poder Legislativo;

\textbf{XIII - examinar, em qualquer órgão dos Poderes Judiciário e Legislativo, ou da Administração Pública em geral, autos de processos findos ou em andamento, mesmo sem procuração, quando não estejam sujeitos a sigilo, assegurada a obtenção de cópias, podendo tomar apontamentos;}

XIV - examinar em qualquer repartição policial, mesmo sem procuração, autos de flagrante e de inquérito, findos ou em andamento, ainda que conclusos à autoridade, podendo copiar peças e tomar apontamentos;

XIV - examinar, em qualquer instituição responsável por conduzir investigação, mesmo sem procuração, autos de flagrante e de investigações de qualquer natureza, findos ou em andamento, ainda que conclusos à autoridade, podendo copiar peças e tomar apontamentos, em meio físico ou digital;         (Redação dada pela Lei nº 13.245, de 2016)

XV - ter vista dos processos judiciais ou administrativos de qualquer natureza, em cartório ou na repartição competente, ou retirá-los pelos prazos legais;

XVI - retirar autos de processos findos, mesmo sem procuração, pelo prazo de dez dias;

XVII - ser publicamente desagravado, quando ofendido no exercício da profissão ou em razão dela;

XVIII - usar os símbolos privativos da profissão de advogado;

XIX - recusar-se a depor como testemunha em processo no qual funcionou ou deva funcionar, ou sobre fato relacionado com pessoa de quem seja ou foi advogado, mesmo quando autorizado ou solicitado pelo constituinte, bem como sobre fato que constitua sigilo profissional;

XX - retirar-se do recinto onde se encontre aguardando pregão para ato judicial, após trinta minutos do horário designado e ao qual ainda não tenha comparecido a autoridade que deva presidir a ele, mediante comunicação protocolizada em juízo.

XXI - assistir a seus clientes investigados durante a apuração de infrações, sob pena de nulidade absoluta do respectivo interrogatório ou depoimento e, subsequentemente, de todos os elementos investigatórios e probatórios dele decorrentes ou derivados, direta ou indiretamente, podendo, inclusive, no curso da respectiva apuração:         (Incluído pela Lei nº 13.245, de 2016)

a) apresentar razões e quesitos;         (Incluído pela Lei nº 13.245, de 2016)

b) (VETADO).         (Incluído pela Lei nº 13.245, de 2016)

§ 1º Não se aplica o disposto nos incisos XV e XVI:

1) aos processos sob regime de segredo de justiça;

2) quando existirem nos autos documentos originais de difícil restauração ou ocorrer circunstância relevante que justifique a permanência dos autos no cartório, secretaria ou repartição, reconhecida pela autoridade em despacho motivado, proferido de ofício, mediante representação ou a requerimento da parte interessada;

3) até o encerramento do processo, ao advogado que houver deixado de devolver os respectivos autos no prazo legal, e só o fizer depois de intimado.

§ 2º O advogado tem imunidade profissional, não constituindo injúria, difamação ou desacato puníveis qualquer manifestação de sua parte, no exercício de sua atividade, em juízo ou fora dele, sem prejuízo das sanções disciplinares perante a OAB, pelos excessos que cometer.        (Vide ADIN 1.127-8)

§ 3º O advogado somente poderá ser preso em flagrante, por motivo de exercício da profissão, em caso de crime inafiançável, observado o disposto no inciso IV deste artigo.

§ 4º O Poder Judiciário e o Poder Executivo devem instalar, em todos os juizados, fóruns, tribunais, delegacias de polícia e presídios, salas especiais permanentes para os advogados, com uso e controle assegurados à OAB.     (Vide ADIN 1.127-8)

§ 5º No caso de ofensa a inscrito na OAB, no exercício da profissão ou de cargo ou função de órgão da OAB, o conselho competente deve promover o desagravo público do ofendido, sem prejuízo da responsabilidade criminal em que incorrer o infrator.

§ 6o  Presentes indícios de autoria e materialidade da prática de crime por parte de advogado, a autoridade judiciária competente poderá decretar a quebra da inviolabilidade de que trata o inciso II do caput deste artigo, em decisão motivada, expedindo mandado de busca e apreensão, específico e pormenorizado, a ser cumprido na presença de representante da OAB, sendo, em qualquer hipótese, vedada a utilização dos documentos, das mídias e dos objetos pertencentes a clientes do advogado averiguado, bem como dos demais instrumentos de trabalho que contenham informações sobre clientes.       (Incluído pela Lei nº 11.767, de 2008)

§ 7o  A ressalva constante do § 6o deste artigo não se estende a clientes do advogado averiguado que estejam sendo formalmente investigados como seus partícipes ou co-autores pela prática do mesmo crime que deu causa à quebra da inviolabilidade.       (Incluído pela Lei nº 11.767, de 2008)

§ 8o  (VETADO)       (Incluído pela Lei nº 11.767, de 2008)

§ 9o  (VETADO)       (Incluído pela Lei nº 11.767, de 2008)

§ 10.  Nos autos sujeitos a sigilo, deve o advogado apresentar procuração para o exercício dos direitos de que trata o inciso XIV.          (Incluído pela Lei nº 13.245, de 2016)

§ 11.  No caso previsto no inciso XIV, a autoridade competente poderá delimitar o acesso do advogado aos elementos de prova relacionados a diligências em andamento e ainda não documentados nos autos, quando houver risco de comprometimento da eficiência, da eficácia ou da finalidade das diligências.          (Incluído pela Lei nº 13.245, de 2016)

§ 12.  A inobservância aos direitos estabelecidos no inciso XIV, o fornecimento incompleto de autos ou o fornecimento de autos em que houve a retirada de peças já incluídas no caderno investigativo implicará responsabilização criminal e funcional por abuso de autoridade do responsável que impedir o acesso do advogado com o intuito de prejudicar o exercício da defesa, sem prejuízo do direito subjetivo do advogado de requerer acesso aos autos ao juiz competente.         (Incluído pela Lei nº 13.245, de 2016)

\noindent\makebox[\linewidth]{\rule{\paperwidth}{0.4pt}}


\section*{Terceira leva de questões analisadas - 11	/08/2017} 

\section{Questão 1 do X° Exame Unificado da Ordem (2013)}

\subsection{Enunciado}

O advogado João, que também é formado em Comunicação 
Social, atua nas duas profissões, possuindo uma coluna onde 
apresenta noticias jurídicas, com informações sobre atividades 
policiais, forenses ou vinculadas ao Ministério Público. 
Semanalmente inclui, nos seus comentários, alguns em forma 
de poesia, suas alegações forenses e os resultados dos 
processos sob sua responsabilidade, divulgando, com isso, seu 
trabalho como advogado. 
 
À luz das normas estatutárias, assinale a afirmativa correta. 
 
OPTIONS

A) A divulgação de notícias, como aventado no enunciado, 
constitui um direito do advogado em dar publicidade aos 
seus processos 

\textbf{B:CORRECT) Nos termos das regras que caracterizam as infrações 
disciplinares está delineada a de publicação desnecessária 
e habitual de alegações forenses ou causas pendentes. }

C) Diante das novas mídias que também atingem a advocacia, 
o advogado pode utilizar-se dos meios ofertados para a 
divulgação de seu trabalho. 

D) A situação caracteriza o chamado desvio da função de 
advogado, com o prejuízo à imagem dos clientes pela 
divulgação. 

\subsection{Fundamento}

O fundamento está no inciso XIII do artigo 34 da Lei 8906, o Estatuto da OAB:

Art. 34. Constitui infração disciplinar:

I - exercer a profissão, quando impedido de fazê-lo, ou facilitar, por qualquer meio, o seu exercício aos não inscritos, proibidos ou impedidos;

II - manter sociedade profissional fora das normas e preceitos estabelecidos nesta lei;

III - valer-se de agenciador de causas, mediante participação nos honorários a receber;

IV - angariar ou captar causas, com ou sem a intervenção de terceiros;

V - assinar qualquer escrito destinado a processo judicial ou para fim extrajudicial que não tenha feito, ou em que não tenha     colaborado;

VI - advogar contra literal disposição de lei, presumindo-se a boa-fé quando fundamentado na inconstitucionalidade, na injustiça da lei ou em pronunciamento judicial anterior;

VII - violar, sem justa causa, sigilo profissional;

\textbf{VIII - estabelecer entendimento com a parte adversa sem autorização do cliente ou ciência do advogado contrário;}

IX - prejudicar, por culpa grave, interesse confiado ao seu patrocínio;

X - acarretar, conscientemente, por ato próprio, a anulação ou a nulidade do processo em que funcione;

XI - abandonar a causa sem justo motivo ou antes de decorridos dez dias da comunicação da renúncia;

XII - recusar-se a prestar, sem justo motivo, assistência jurídica, quando nomeado em virtude de impossibilidade da Defensoria Pública;

XIII - fazer publicar na imprensa, desnecessária e habitualmente, alegações forenses ou relativas a causas pendentes;

XIV - deturpar o teor de dispositivo de lei, de citação doutrinária ou de julgado, bem como de depoimentos, documentos e alegações da parte contrária, para confundir o adversário ou iludir o juiz da causa;

XV - fazer, em nome do constituinte, sem autorização escrita deste, imputação a terceiro de fato definido como crime;

XVI - deixar de cumprir, no prazo estabelecido, determinação emanada do órgão ou de autoridade da Ordem, em matéria da competência desta, depois de regularmente notificado;

XVII - prestar concurso a clientes ou a terceiros para realização de ato contrário à lei ou destinado a fraudá-la;

XVIII - solicitar ou receber de constituinte qualquer importância para aplicação ilícita ou desonesta;

XIX - receber valores, da parte contrária ou de terceiro, relacionados com o objeto do mandato, sem expressa autorização do constituinte;

XX - locupletar-se, por qualquer forma, à custa do cliente ou da parte adversa, por si ou interposta pessoa;

XXI - recusar-se, injustificadamente, a prestar contas ao cliente de quantias recebidas dele ou de terceiros por conta dele;

XXII - reter, abusivamente, ou extraviar autos recebidos com vista ou em confiança;

XXIII - deixar de pagar as contribuições, multas e preços de serviços devidos à OAB, depois de regularmente notificado a fazê-lo;

XXIV - incidir em erros reiterados que evidenciem inépcia profissional;

XXV - manter conduta incompatível com a advocacia;

XXVI - fazer falsa prova de qualquer dos requisitos para inscrição na OAB;

XXVII - tornar-se moralmente inidôneo para o exercício da advocacia;

XXVIII - praticar crime infamante;

XXIX - praticar, o estagiário, ato excedente de sua habilitação.

Parágrafo único. Inclui-se na conduta incompatível:

a) prática reiterada de jogo de azar, não autorizado por lei;

b) incontinência pública e escandalosa;

c) embriaguez ou toxicomania habituais.


\section{Questão 2 do X° Exame Unificado da Ordem (2013)}

\subsection{Enunciado}

O advogado Mário pertence aos quadros da sociedade de 
economia mista controlada pelo Estado W, na qual chefia o 
Departamento Jurídico. Não existe óbice para a prestação de 
serviços de advocacia privada, o que ocorre no escritório que 
possui no centro da capital do Estado, em horário diverso do 
expediente na empresa. Um dos seus clientes realiza contrato 
para que Mário aponha o seu visto em ato constitutivo de 
pessoa jurídica, em Junta Comercial cuja sede está localizada 
na capital do Estado W.  
 
Observado tal relato, consoante as normas do Regulamento 
Geral do Estatuto da Advocacia e da OAB, assinale a afirmativa 
correta. 
 
OPTIONS

A) As circunstâncias indicam que não existe óbice para a 
aposição do visto nos referidos atos. 

B) O fato de chefiar Departamento Jurídico de empresa, seja 
de que natureza for, constitui elemento impeditivo da 
aposição do visto. 

C) O exercício da advocacia no local da sede da Junta 
Comercial é impeditivo para a aposição do visto. 

\textbf{D:CORRECT) A atuação em sociedade de economia mista estadual 
impede a aposição do visto contratado. }

\subsection{Fundamento}

Nesse caso, a resposta está baseada no parágrafo único do Artigo 2° do Regulamento Geral do Estatuto da Advocacia e da OAB.

Perceba que é importante, para responder a questão, perceber que o enunciado se refere a sociedade de economia mista, sendo que a lei diz entidade da administração pública indireta. Para acertar o candidato precisava saber que a sociedade de economia mista é um dos tipos de instituição que compoe a administração pública indireta.

Apenas para ilustrar a questão, a Petrobras e o Banco do Brasil são exemplos de sociedades de economia mista, já que tem como sócios o Estados e agentes da iniciativa privada.

Existem outras instituição que também compõem a administração pública indireta, como as autarquias e as empresas públlicas. Texto do regulamento:

Art. 2º O visto do advogado em atos constitutivos de pessoas jurídicas, indispensável ao registro e arquivamento nos órgãos competentes, deve resultar da efetiva constatação, pelo profissional que os examinar, de que os respectivos instrumentos preenchem as exigências legais pertinentes. 

\textbf{Parágrafo único. Estão impedidos de exercer o ato de advocacia referido neste artigo os advogados que  prestem  serviços  a  órgãos  ou  entidades  da  Administração  Pública  direta  ou  indireta,  da 
unidade federativa a que se vincule a Junta Comercial, ou a quaisquer repartições administrativas competentes para o mencionado registro. }


\section{Questão 3 do X° Exame Unificado da Ordem (2013)}

\subsection{Enunciado}

dos Advogados do Brasil, veio a ser indiciado por força de 
investigação proposta em face de um dos seus inúmeros 
clientes, não tendo o causídico participado de qualquer ato 
ilícito, mas apenas como advogado. Veio a saber que seu 
nome fora incluído por força de exercício considerado 
exacerbado de sua atividade advocatícia. Contratou advogado 
para a sua defesa no inquérito criminal e postulou assistência 
à Ordem dos Advogados do Brasil por entender feridas suas 
prerrogativas profissionais.  
 
Observado tal relato, consoante as normas do Regulamento 
Geral do Estatuto da Advocacia e da OAB, assinale a afirmativa 
correta. 
 
OPTIONS

A) Ao contratar advogado para a defesa da sua pretensão, 
não mais cabe à Ordem dos Advogados interferir no 
processo para salvaguardar eventuais prerrogativas 
feridas. 

\textbf{B:CORRECT) A atuação da Ordem dos Advogados na defesa das 
prerrogativas profissionais implicará a assistência de 
representante da instituição, mesmo com defensor 
constituído. }

C) A assistência da Ordem dos Advogados está restrita a 
processos judiciais ou administrativos, mas não a 
inquéritos. 

D) A postulação de assistência deve ser examinada pelo 
Conselho Federal da Ordem dos Advogados que pode 
autorizar ou não essa atividade. 

\subsection{Fundamento}

A resposta está no Artigo 16 do Regulamento da OAB

Art.  16.  Sem  prejuízo  da  atuação  de  seu  defensor,  contará  o  advogado  com  a  assistência  de  representante da OAB nos inquéritos policiais ou nas ações penais em que figurar como indiciado, acusado ou ofendido, sempre que o fato a ele imputado decorrer do exercício da profissão ou a este vincular-se.


\section{Questão 4 do X° Exame Unificado da Ordem (2013)}

\subsection{Enunciado}

Nos termos do Estatuto da Advocacia existe a previsão de 
pagamento de honorários advocatícios. Assinale a afirmativa 
que indica como deve ocorrer o pagamento, quando não 
houver estipulação em contrário.  
 
OPTIONS

A) Metade no início e o restante parcelado em duas vezes. 

\textbf{B:CORRECT) Um terço no inicio, um terço até a decisão de primeira instância e um terço ao final. }

C) Dez por cento no início, vinte por cento na sentença e o 
restante após o trânsito em julgado. 

D) Cinquenta por cento no início, trinta por cento até decisão 
de primeiro grau e o restante após o recurso, se existir. 

\subsection{Fundamento}

O embasamento está na lei 8906, no artigo 22, parágrafo terceiro:

Art. 22. A prestação de serviço profissional assegura aos inscritos na OAB o direito aos honorários convencionados, aos fixados por arbitramento judicial e aos de sucumbência.

§ 1º O advogado, quando indicado para patrocinar causa de juridicamente necessitado, no caso de impossibilidade da Defensoria Pública no local da prestação de serviço, tem direito aos honorários fixados pelo juiz, segundo tabela organizada pelo Conselho Seccional 
da OAB, e pagos pelo Estado.

§ 2º Na falta de estipulação ou de acordo, os honorários são fixados por arbitramento judicial, em remuneração compatível com o trabalho e o valor econômico da questão, não podendo ser inferiores aos estabelecidos na tabela organizada pelo Conselho Seccional da OAB.

\textbf{§ 3º Salvo estipulação em contrário, um terço dos honorários é devido no início do serviço, outro terço até a decisão de primeira instância e o restante no final.}

§ 4º Se o advogado fizer juntar aos autos o seu contrato de honorários antes de expedir-se o mandado de levantamento ou precatório, o juiz deve determinar que lhe sejam pagos diretamente, por dedução da quantia a ser recebida pelo constituinte, salvo se este provar que já os pagou.

§ 5º O disposto neste artigo não se aplica quando se tratar de mandato outorgado por advogado para defesa em processo oriundo de ato ou omissão praticada no exercício da profissão.

\section{Questão 1 do XIX° Exame Unificado da Ordem (2016)}

\subsection{Enunciado}

Alexandre, advogado que exerce a profissão há muitos anos, é 
conhecido por suas atitudes corajosas, sendo respeitado pelos 
seus clientes e pelas autoridades com quem se relaciona por 
questões profissionais. Comentando sua atuação profissional, 
ele foi inquirido, por um dos seus filhos, se não deveria 
recusar a defesa de um indivíduo considerado impopular, bem 
como se não deveria ser mais obediente às autoridades, 
diante da possibilidade de retaliação. 
 
Sobre o caso apresentado, observadas as regras do Estatuto 
da OAB, assinale a opção correta indicada ao filho do 
advogado citado. 
 
OPTIONS

A) O advogado Alexandre deve recusar a defesa de cliente 
cuja atividade seja impopular. 

B) O temor à autoridade pode levar à negativa de prestação 
do serviço advocatício por Alexandre. 

C) As causas impopulares aceitas por Alexandre devem vir 
sempre acompanhadas de apoio da Seccional da OAB. 

\textbf{D:CORRECT) Nenhum receio de desagradar uma autoridade deterá o 
advogado Alexandre. }

\subsection{Fundamento}

Art 31, parágrafo §2 da Lei 8906:

Art. 31. O advogado deve proceder de forma que o torne merecedor de respeito e que contribua para o prestígio da classe e da advocacia.

§ 1º O advogado, no exercício da profissão, deve manter independência em qualquer circunstância.

\textbf{§ 2º Nenhum receio de desagradar a magistrado ou a qualquer autoridade, nem de incorrer em impopularidade, deve deter o advogado no exercício da profissão}


\section{Questão 2 do XIX° Exame Unificado da Ordem (2016)}

\subsection{Enunciado}


O advogado Carlos dirigiu-se a uma Delegacia de Polícia para 
tentar obter cópia de autos de inquérito no âmbito do qual 
seu cliente havia sido intimado para prestar esclarecimentos. 

No entanto, a vista dos autos foi negada pela autoridade 
policial, ao fundamento de que os autos estavam sob segredo 
de Justiça. Mesmo após Carlos ter apresentado procuração de 
seu cliente, afirmou o Delegado que, uma vez que o juiz havia 
decretado sigilo nos autos, a vista somente seria permitida 
com autorização judicial. 
 
Nos termos do Estatuto da Advocacia, é correto afirmar que 
 
OPTIONS

A) Carlos pode ter acesso aos autos de qualquer inquérito, 
mesmo sem procuração. 

\textbf{B:CORRECT) Carlos pode ter acesso aos autos de inquéritos sob segredo de Justiça, desde que esteja munido de procuração do 
investigado. }

C) em caso de inquérito sob segredo de Justiça, apenas o 
magistrado que decretou o sigilo poderá afastar 
parcialmente o sigilo, autorizando o acesso aos autos pelo 
advogado Carlos. 

D) o segredo de Justiça de inquéritos em andamento é 
oponível ao advogado Carlos, mesmo munido de 
procuração. 


\subsection{Fundamento}

Essa questão tem como justificativa duas partes do art. 7º do Estatuto da Advocacia (Lei 8906). O inciso XIV e o parágrafo 10. Perceba que o próprio parágrafo § 10 cita o inciso relacionado: 

Art. 7º São direitos do advogado: 

XIV – examinar, em qualquer instituição re sponsável por conduzir investigação, mesmo sem procuração, autos de flagrante e de investigações de qualquer natureza, findos ou em andamento, ainda que conc lusos à autoridade, podendo copiar peças e tomar apontamentos, em meio físico ou digital; 

Agora, muita atenção para o §10:

§10: Nos autos sujeitos a sigilo, deve o advogado 
apresentar procuração para o exercício dos direitos de que trata o inciso XIV. 

Veja que o advogado pode ter acesso aos autos de inquéritos sob segredo de Justiça, mas para isso, é necessário a procuração do investigado. 


\section{Questão 3 do XIX° Exame Unificado da Ordem (2016)}

\subsection{Enunciado}

Tício, presidente de determinada Subseção da OAB, valendo-
se da disciplina do Art. 50 da Lei Federal nº 8.906/94 (Estatuto 
da OAB), pretende requisitar, ao cartório de certa Vara de 
Fazenda Pública, cópias de peças dos autos de um processo 
judicial que não estão cobertas pelo sigilo. Assim, analisou o 
entendimento jurisprudencial consolidado no Supremo 
Tribunal Federal sobre o tema, a fim de apurar a possibilidade 
da requisição, bem como, caso positivo, a necessidade de 
motivação e pagamento dos custos respectivos. 
 
Diante da situação narrada, Tício estará correto ao concluir 
que 

OPTIONS

A) não dispõe de tal prerrogativa, pois o citado dispositivo 
legal foi declarado inconstitucional pelo Supremo Tribunal 
Federal, uma vez que compete privativamente aos 
tribunais organizar as secretarias e cartórios judiciais, não 
se sujeitando a requisições da OAB, por expressa disciplina 
constitucional. 

B) pode realizar tal requisição, pois o citado dispositivo legal 
foi declarado constitucional pelo Supremo Tribunal 
Federal, independentemente de motivação e pagamento 
dos respectivos custos. 

\textbf{C:CORRECT) pode realizar tal requisição, pois o Supremo Tribunal Federal, em sede de controle de constitucionalidade, 
assegurou-a, desde que acompanhada de motivação 
compatível com as finalidades da Lei nº 8.906/94 e o 
pagamento dos respectivos custos.} 

D) não dispõe de tal prerrogativa, pois ao citado dispositivo 
legal foi conferida, pelo Supremo Tribunal Federal, 
interpretação conforme a Constituição Federal para excluir 
os presidentes de Subseções, garantindo a requisição 
apenas aos Presidentes do Conselho Federal da OAB e dos 
Conselhos Seccionais, desde que motivada. 

\subsection{Fundamento}

Essa questão é interessante já que a resposta NÃO está apenas na lei. Para respondê-la, é necessário conhecer a jurisprudência do Supremo Tribunal Federal (STF). 

O art. 50 da Lei 8906 define: 

Art. 50. Para os fins desta lei, os Presidentes dos Conselhos da OAB e das Subseções podem requisitar cópias de peças de autos e documentos a qualquer tribunal, magistrado, cartório e órgão da Administração Pública direta, indireta e fundacional.       

Esse dispositivo permite que os presidentes dos Conselhos da OAB e das Subseções possam requisitar cópias de peças de autos e documentos a qualquer tribunal, magistrado, cartório e órgão da Administração Pública direta, indireta e fundacional. 

\textbf{O Plenário do STF, ao apreciar a ADI 1127, julgou parcialmente procedente a ação nesse ponto para dar interpretação conforme a Constituição, no sentido de compreender a expressão ''requisitar'' como dependente de motivação, compatibilização com as finalidades da lei e atendimento de custos desta requisição, ressalvados os documentos cobertos por sigilo.}

\section{Questão 5 do XIX° Exame Unificado da Ordem (2016)}

\subsection{Enunciado}

Daniel contratou a advogada Beatriz para ajuizar ação em face 
de seu vizinho Théo, buscando o ressarcimento de danos 
causados em razão de uma obra indevida no condomínio. No 
curso do processo, Beatriz substabeleceu o mandato a Ana, 
com reserva de poderes. Sentenciado o feito e julgado 
procedente o pedido de Daniel, o juiz condenou Théo ao 
pagamento de honorários sucumbenciais. 
 
Com base na hipótese apresentada, assinale a afirmativa 
correta. 

OPTIONS

A) Ana poderá promover a execução dos honorários 
sucumbenciais nos mesmos autos judiciais, se assim lhe 
convier, independentemente da intervenção de Beatriz. 

B) Ana e Beatriz poderão promover a execução dos 
honorários sucumbenciais, isoladamente ou em conjunto, 
mas devem fazê-lo em processo autônomo. 

\textbf{C:CORRECT) Ana poderá promover a execução dos honorários 
sucumbenciais nos mesmos autos, se assim lhe convier, 
mas dependerá da intervenção de Beatriz. }

D) Ana não terá direito ao recebimento de honorários 
sucumbenciais, cabendo-lhe executar Beatriz pelos valores 
que lhe sejam devidos, caso não haja o adimplemento 
voluntário. 

\subsection{Fundamento}

Essa é outra questão interessante, já que irá envolver dois artigos conjulgados. A justificativa da letra C está na interpretação conjunta do artigo 26 e do §1° do artigo 24 da Lei 8906:

\textbf{Art. 26. O advogado substabelecido, com reserva de poderes, não pode cobrar honorários sem a intervenção daquele que lhe conferiu o substabelecimento. }
	
Art. 24. A decisão judicial que fixar ou arbitrar honorários e o contrato escrito que os estipular são títulos executivos e constituem crédito privilegiado na falência, concordata, concurso de credores, insolvência civil e liquidação extrajudicial. 

\textbf{§ 1º A execução dos honorários pode ser promovida nos mesmos autos da ação em que tenha atuado o advogado, se assim lhe convier.}


\subsection{Observação}

Nessa amostra, foi bem raro uma questão exigir dois artigos conjugados para ser o fundamento. Merece destaque na análise.


\section{Questão 6 do XIX° Exame Unificado da Ordem (2016)}

\subsection{Enunciado}

Victor nasceu no Estado do Rio de Janeiro e formou-se em 
Direito no Estado de São Paulo. Posteriormente, passou a 
residir, e pretende atuar profissionalmente como advogado, 
em Fortaleza, Ceará. Porém, em razão de seus contatos no Rio 
de Janeiro, foi convidado a intervir também em feitos judiciais 
em favor de clientes nesse Estado, cabendo-lhe patrocinar seis 
causas no ano de 2015.  
 
Diante do exposto, assinale a opção correta. 
 
OPTIONS

A) A inscrição principal de Victor deve ser realizada no 
Conselho Seccional de São Paulo, já que a inscrição 
principal do advogado é feita no Conselho Seccional em 
cujo território se localize seu curso jurídico. Além da 
principal, Victor terá a faculdade de promover sua 
inscrição suplementar nos Conselhos Seccionais do Ceará e 
do Rio de Janeiro, onde pretende exercer a profissão. 

B) A inscrição principal de Victor deve ser realizada no 
Conselho Seccional do Rio de Janeiro, pois o Estatuto da 
OAB determina que esta seja promovida no Conselho 
Seccional em cujo território o advogado exercer 
intervenção judicial que exceda três causas por ano. Além 
da principal, Victor poderá promover sua inscrição 
suplementar nos Conselhos Seccionais do Ceará e de São 
Paulo. 

C) A inscrição principal de Victor deve ser realizada no 
Conselho Seccional do Ceará. Isso porque a inscrição 
principal do advogado deve ser feita no Conselho Seccional 
em cujo território pretende estabelecer o seu domicílio 
profissional. A promoção de inscrição suplementar no 
Conselho Seccional do Rio de Janeiro será facultativa, pois 
as intervenções judiciais pontuais, como as causas em que 
Victor atuará, não configuram habitualidade no exercício 
da profissão. 

\textbf{D:CORRECT) A inscrição principal de Victor deve ser realizada no Conselho Seccional do Ceará. Afinal, a inscrição principal 
do advogado deve ser feita no Conselho Seccional em cujo 
território ele pretende estabelecer o seu domicílio 
profissional. Além da principal, Victor deverá promover a 
inscrição suplementar no Conselho Seccional do Rio de 
Janeiro, já que esta é exigida diante de intervenção judicial 
que exceda cinco causas por ano. }

\subsection{Fundamento}

A resposta está embada no arito 10 §2° da Lei 8906, o Estauto da OAB:

Art. 10. A inscrição principal do advogado deve ser feita no Conselho Seccional em cujo território pretende estabelecer o seu domicílio profissional, na forma do regulamento geral.

§ 1º Considera-se domicílio profissional a sede principal da atividade de advocacia, prevalecendo, na dúvida, o domicílio da pessoa física do advogado.

\textbf{§ 2º Além da principal, o advogado deve promover a inscrição suplementar nos Conselhos Seccionais em cujos territórios passar a exercer habitualmente a profissão considerando-se habitualidade a intervenção judicial que exceder de cinco causas por ano.}

§ 3º No caso de mudança efetiva de domicílio profissional para outra unidade federativa, deve o advogado requerer a transferência de sua inscrição para o Conselho Seccional correspondente.

§ 4º O Conselho Seccional deve suspender o pedido de transferência ou de inscrição suplementar, ao verificar a existência de vício ou ilegalidade na inscrição principal, contra ela representando ao Conselho Federal.


\section{Questão 7 do XIX° Exame Unificado da Ordem (2016)}

\subsection{Enunciado}

Os jovens Rodrigo, 30 anos, e Bibiana, 35 anos, devidamente 
inscritos em certa seccional da OAB, desejam candidatar-se, 
pela primeira vez, a cargos de diretoria do Conselho Seccional 
respectivo. Rodrigo está regularmente inscrito na referida 
seccional da OAB há seis anos, sendo dois anos como 
estagiário. Bibiana, por sua vez, exerceu regularmente a 
profissão por três anos, após a conclusão do curso de Direito. 
Contudo, afastou-se por dois anos e retornou à advocacia há 
um ano. Ambos não exercem funções incompatíveis com a 
advocacia, ou cargos exoneráveis ad nutum. Tampouco 
integram listas para provimento de cargos em tribunais ou 
ostentam condenação por infração disciplinar. Bibiana e 
Rodrigo estão em dia com suas anuidades. 
Considerando a situação narrada, assinale a afirmativa 
correta. 
 
OPTIONS

A) Apenas Bibiana preenche as condições de elegibilidade 
para os cargos. 

B) Apenas Rodrigo preenche as condições de elegibilidade 
para os cargos. 

C) Bibiana e Rodrigo preenchem as condições de elegibilidade 
para os cargos. 

\textbf{D:CORRECT) Nenhum dos dois advogados preenche as condições de 
elegibilidade para os cargos.}

\subsection{Fundamento}

A resposta está embasada no Artigo 131-A do Regulamento da OAB. Esse é um documento diferente do Estatuto da OAB e que, como disse antes, não é uma lei, mas uma norma prevista na Lei 8906. Norma e lei são coisas diferentes. Toda lei é uma norma. Mas nem toda norma é lei, feita pelo executivo ou pelo legislativo. Nenhum dos dois possui 5 anos contínuos.

\textbf{Art. 131 - A. São condições de elegibilidade: ser o candidato advogado inscrito na Seccional, com inscrição principal ou suplementar, em efetivo exercício há mais de 05 (cinco) anos, e estar em dia com as anuidades na data de protocolo do pedido de registro de candidatura, considerando-se regulares aqueles que parcelaram seus débitos e estão adimplentes com a quitação das parcelas. }

§ 1º O candidato deverá comprovar sua adimplência junto à OAB por meio da apresentação de certidão da Seccional onde é candidato. 

§ 2º Sendo o candidato inscrito em várias Seccionais, deverá, ainda, quando da inscrição da chapa na  qual  concorrer,  declarar,  sob  a  sua  responsabilidade  e  sob  as  penas  legais,  que  se  encontra adimplente com todas elas.

\textbf{§  3º  O  período  de  05  (cinco)  anos  estabelecido  no caput deste  artigo  é  o  que  antecede imediatamente a data da posse, computado continuamente.}
	


\noindent\makebox[\linewidth]{\rule{\paperwidth}{0.4pt}}

\section*{Quarta leva de questões analisadas - 14/09/2017} 


\section{Questão 2 do XVIII° Exame Unificado da Ordem (2015)}

\subsection{Enunciado}
Os advogados criminalistas X e Y atuavam em diversas ações 
penais e inquéritos em favor de um grupo de pessoas 
acusadas de pertencer a determinada organização criminosa, 
supostamente destinada ao tráfico de drogas. Ao perceber 
que não havia outros meios disponíveis para a obtenção de 
provas contra os investigados, o juiz, no âmbito de um dos 
inquéritos instaurados para investigar o grupo, atendendo à 
representação da autoridade policial e considerando 
manifestação favorável do Ministério Público, determinou o 
afastamento do sigilo telefônico dos advogados constituídos 
nos autos dos aludidos procedimentos, embora não houvesse 
indícios da prática de crimes por estes últimos. As conversas 
entre os investigados e seus advogados, bem como aquelas 
havidas entre os advogados X e Y, foram posteriormente 
usadas para fundamentar a denúncia oferecida contra seus 
clientes. 

Considerando-se a hipótese apresentada, assinale a afirmativa 
correta. 
 
OPTIONS

A) A prova é lícita, pois não havia outro meio disponível para a obtenção de provas. 

B) A prova é lícita, pois tratava-se de investigação de prática de crime cometido no âmbito de organização criminosa. 

C) Considerando que não havia outro meio disponível para a 
obtenção de provas, bem como que se tratava de 
investigação de prática de crime cometido no âmbito de 
organização criminosa, é ilícita a prova obtida a partir dos 
diálogos havidos entre os advogados e seus clientes. É, no 
entanto, lícita a prova obtida a partir dos diálogos havidos 
entre os advogados X e Y. 

\textbf{D:CORRECT) A prova é ilícita, uma vez que as comunicações telefônicas do advogado são invioláveis quando disserem respeito ao exercício da profissão, bem como se não houver indícios da prática de crime pelo advogado.} 

\subsection{Fundamento}

Se não havia indícios da prática de crimes pelos advogados, o sigilo não poderia ter sido quebrado. Veja a redação da norma:

Art. 7o São direitos do advogado:

II – a inviolabilidade de seu escritório ou local de trabalho, bem como de seus instrumentos de trabalho, de sua correspondência escrita, eletrônica, telefônica e telemática, desde que relativas ao exercício da advocacia;(Redação dada pela Lei no 11.767, de 2008)

§ 6o Presentes indícios de autoria e materialidade da prática de crime por parte de advogado, a autoridade judiciária competente poderá decretar a quebra da inviolabilidade de que trata o inciso II do caput deste artigo, em decisão motivada, expedindo mandado de busca e apreensão, específico e pormenorizado, a ser cumprido na presença de representante da OAB, sendo, em qualquer hipótese, vedada a utilização dos documentos, das mídias e dos objetos pertencentes a clientes do advogado averiguado, bem como dos demais instrumentos de trabalho que contenham informações sobre clientes. (Incluído pela Lei no 11.767, de 2008)

\subsection{Algoritmo}

O algoritmo acertou o artigo 7.
\lstset{
    string=[s]{"}{"},
    tabsize=3,
    frame=shadowbox
    stringstyle=\color{blue},
    comment=[l]{:},
    commentstyle=\color{black},
     rulesepcolor=\color{gray},
    xleftmargin=20pt,
    framexleftmargin=15pt,
    keywordstyle=\color{blue}\bf,
    commentstyle=\color{OliveGreen},
    stringstyle=\color{red},
    numbers=left,
    numberstyle=\tiny,
    numbersep=5pt,
    breaklines=true,
    showstringspaces=false,
    basicstyle=\footnotesize,
    extendedchars=true,
    literate={º}{{$^o$}}1 {õ}{{\~o}}1 {ã}{{\~a}}1 {ç}{{\'{c}}}1 {á}{{\'a}}1 {ó}{{\'o}}1 {í}{{\'i}}1 {§}{{\S}}1 {â}{{\^a}}1,
    emph={food,name,price},emphstyle={\color{magenta}}
}

\begin{lstlisting}
       "OAB:2015-18|Q2|ans:D|just:.": {
            "A": [
                24.580946668021195,
                [
                    "2",
                    "urn:lex:br:federal:lei:1994-07-04;8906art7",
                    "A"
                ]
            ],
            "B": [
                19.563478896527062,
                [
                    "2",
                    "urn:lex:br:federal:lei:1994-07-04;8906art7",
                    "B"
                ]
            ],
            "C": [
                17.140894080004543,
                [
                    "2",
                    "urn:lex:br:federal:lei:1994-07-04;8906art7",
                    "C"
                ]
            ],
            "D": [
                14.328697302210664,
                [
                    "2",
                    "urn:lex:br:federal:lei:1994-07-04;8906art7",
                    "D"
                ]
            ]
        },

\end{lstlisting}



\section{Questão 3 do XVIII° Exame Unificado da Ordem (2015)}

\subsection{Enunciado}

A advogada Ana retirou de cartório os autos de determinado 
processo de conhecimento em que representava a parte ré, 
para apresentar contestação. Protocolou a petição 
tempestivamente, mas deixou de devolver os autos em 
seguida por esquecimento, só o fazendo após ficar pouco mais 
de um mês com os autos em seu poder. Ao perceber que Ana 
não devolvera os autos imediatamente após cumprir o prazo, 
o magistrado exarou despacho pelo qual a advogada foi 
proibida de retirar novamente os autos do cartório em carga, 
até o final do processo. 
 
Nos termos do Estatuto da Advocacia, deve-se assentar 
quanto à sanção disciplinar que 
 
OPTIONS

\textbf{A:CORRECT) não se aplica porque Ana não chegou a ser intimada a devolver os autos. }

B) não se aplica porque Ana ficou menos de três meses com 
os autos em seu poder. 

C) aplica-se porque Ana reteve abusivamente os autos em 
seu poder. 

D) aplica-se porque Ana não poderia ter retirado os autos de 
cartório para cumprir o prazo assinalado para contestação. 

\subsection{Fundamento}

Com relação à retenção dos autos indevidamente, veja o que dispõe a Lei 8906 da OAB:

Art. 34. Constitui infração disciplinar:

XXII – reter, abusivamente, ou extraviar autos recebidos com vista ou em confiança;

Contudo, a parte final do ponto 3 do § 1o do art. 7o informa que somente será vedado o direito a retirar os autos se o advogado só devolveu os autos depois de intimado.

Art. 7o:

§ 1o Não se aplica o disposto nos incisos XV e XVI:
(...)

3) até o encerramento do processo, ao advogado que houver deixado de devolver os respectivos autos no prazo legal, e só o fizer depois de intimado.

\subsection{Algoritmo}

O algoritmo acertou parcialmente. A resposta envolvia também outro artigo, além do 7°. O artigo 7° é bem grande. Seria importante indicar o inciso.

\begin{lstlisting}
        "OAB:2015-18|Q3|ans:A|just:.": {
            "A": [
                11.272877977159153,
                [
                    "3",
                    "urn:lex:br:federal:lei:1994-07-04;8906art7",
                    "A"
                ]
            ],
            "B": [
                12.805056906471412,
                [
                    "3",
                    "urn:lex:br:federal:lei:1994-07-04;8906art7",
                    "B"
                ]
            ],
            "C": [
                12.65073878501901,
                [
                    "3",
                    "urn:lex:br:federal:lei:1994-07-04;8906art7",
                    "C"
                ]
            ],
            "D": [
                8.359817802429868,
                [
                    "3",
                    "urn:lex:br:federal:lei:1994-07-04;8906art50",
                    "D"
                ]
            ]
        },

\end{lstlisting}

\section{Questão 4 do XVIII° Exame Unificado da Ordem (2015)}

\subsection{Enunciado}

Fernanda, estudante do 8º período de Direito, requereu 
inscrição junto à Seccional da OAB do estado onde reside. A 
inscrição foi indeferida, em razão de Fernanda ser 
serventuária do Tribunal de Justiça do estado. Fernanda 
recorreu da decisão, alegando que preenche todos os 
requisitos exigidos em lei para a inscrição de estagiário e que o 
exercício de cargo incompatível com a advocacia não impede a 
inscrição do estudante de Direito como estagiário. 

Merece ser revista a decisão que indeferiu a inscrição de 
estagiário de Fernanda? 
 
OPTIONS

A) Sim, pois Fernanda exerce cargo incompatível com a 
advocacia e não com a realização de estágio. 

\textbf{B:CORRECT) Não, pois as incompatibilidades previstas em lei para o exercício da advocacia também devem ser observadas quando do requerimento de inscrição de estagiário.}

C) Sim, pois o cargo de serventuário do Tribunal de Justiça 
não é incompatível com a advocacia, menos ainda com a 
realização de estágio. 

D) Não, pois apenas estudantes do último período do curso 
de Direito podem requerer inscrição como estagiários. 

\subsection{Fundamento}

A resposta envolve dois artigos da Lei 8906:

Art. 9o Para inscrição como estagiário é necessário:

(...)

§ 3o O aluno de curso jurídico que exerça atividade incompatível com a advocacia pode
freqüentar o estágio ministrado pela respectiva instituição de ensino superior, para fins de
aprendizagem, vedada a inscrição na OAB.
Veja que gera a incompatibilidade o exercício de cargo no Poder Judiciário:

Art. 28. A advocacia é incompatível, mesmo em causa própria, com as seguintes
atividades:

(...)

IV – ocupantes de cargos ou funções vinculados direta ou indiretamente a qualquer órgão
do Poder Judiciário e os que exercem serviços notariais e de registro;

\subsection{Algoritmo}

O algoritmo indicou \textbf{incorretamente} o artigo mais importante para a questão. Ele indicou arigo 29 e deveria ter indicado o 28.

\begin{lstlisting}
         "OAB:2015-18|Q4|ans:B|just:.": {
            "A": [
                9.550330921194686,
                [
                    "4",
                    "urn:lex:br:federal:lei:1994-07-04;8906art9",
                    "A"
                ]
            ],
            "B": [
                13.306066812920733,
                [
                    "4",
                    "urn:lex:br:ordem.advogados.brasil;conselho.federal:regulamento.geral:1994-10-16;seq-oab-1art29",
                    "B"
                ]
            ],
            "C": [
                10.243408538251124,
                [
                    "4",
                    "urn:lex:br:federal:lei:1994-07-04;8906art9",
                    "C"
                ]
            ],
            "D": [
                12.41882577998806,
                [
                    "4",
                    "urn:lex:br:federal:lei:1994-07-04;8906art9",
                    "D"
                ]
            ]
        },

\end{lstlisting}

\section{Questão 5 do XVIII° Exame Unificado da Ordem (2015)}

\subsection{Enunciado}

Determinada causa em que se discutia a guarda de dois 
menores estava confiada ao advogado Álvaro, que trabalhava 
sozinho em seu escritório. Aproveitando o período de recesso 
forense e considerando que não teria prazos a cumprir ou atos 
processuais designados durante esse período, Álvaro realizou 
viagem para visitar a família no interior do estado. Alguns dias 
depois de sua partida, ainda durante o período de recesso, 
instalou-se situação que demandaria a tomada de medidas 
urgentes no âmbito da mencionada ação de guarda. O cliente 
de Álvaro, considerando que seu advogado se encontrava fora 
da cidade, procurou outro advogado, Paulo, para que a 
medida judicial necessária fosse tomada, recorrendo-se ao 
plantão judiciário. Paulo não conseguiu falar com Álvaro para 
avisar que atuaria na causa em que este último estava 
constituído, mas aceitou procuração do cliente assim mesmo e 
tomou a providência cabível. 
 
Poderia Paulo ter atuado na causa sem o conhecimento e a 
anuência de Álvaro? 

OPTIONS

\textbf{A:CORRECT) Paulo poderia ter atuado naquela causa apenas para tomar 
a medida urgente cabível.} 

B) Paulo poderia ter atuado na causa, ainda que não 
houvesse providência urgente a tomar, uma vez que o 
advogado constituído estava viajando. 

C) Paulo não poderia ter atuado na causa, pois o advogado 
não pode aceitar procuração de quem já tenha patrono 
constituído, sem prévio conhecimento deste, ainda que 
haja necessidade da tomada de medidas urgentes. 

D) Paulo não poderia ter atuado na causa, pois os prazos 
estavam suspensos durante o recesso. 

\subsection{Fundamento}

Código de Ética e Disciplina da OAB, artigo 11

Art. 11. O advogado não deve aceitar procuração de quem já tenha patrono constituído, sem prévio conhecimento deste, salvo por motivo justo ou para adoção de medidas
judiciais urgentes e inadiáveis.


\subsection{Algoritmo}

O algoritmo \textbf{errou}. Errou, inclusive, a lei aplicável. Ele trocou o regulamento geral com o código de ética da OAB.

\begin{lstlisting}
       "OAB:2015-18|Q5|ans:A|just:.": {
            "A": [
                17.429548132204495,
                [
                    "5",
                    "urn:lex:br:ordem.advogados.brasil;conselho.federal:regulamento.geral:1994-10-16;seq-oab-1art110",
                    "A"
                ]
            ],
            "B": [
                15.615428341430443,
                [
                    "5",
                    "urn:lex:br:ordem.advogados.brasil;conselho.federal:codigo.etica:1995-3-1;seq-oab-1art11",
                    "B"
                ]
            ],
            "C": [
                6.727441105059098,
                [
                    "5",
                    "urn:lex:br:ordem.advogados.brasil;conselho.federal:codigo.etica:1995-3-1;seq-oab-1art11",
                    "C"
                ]
            ],
            "D": [
                11.289647694154032,
                [
                    "5",
                    "urn:lex:br:ordem.advogados.brasil;conselho.federal:regulamento.geral:1994-10-16;seq-oab-1art156-1",
                    "D"
                ]
            ]
        },

\end{lstlisting}


\section{Questão 6 do XVIII° Exame Unificado da Ordem (2015)}

\subsection{Enunciado}

O Presidente de determinada Seccional da OAB recebeu 
representação contra advogado que nela era inscrito por meio 
de missiva anônima, que narrava grave infração disciplinar. 
Considerando a via eleita para a apresentação da representação, foi determinado o arquivamento do expediente, sem instauração de processo disciplinar. Pouco tempo depois, foi publicada matéria jornalística sobre investigação realizada pela Polícia Federal que tinha como objeto a mesma infração disciplinar que havia sido narrada na  missiva anônima e indicando o nome do investigado naquele procedimento inquisitorial. Com base na reportagem, foi determinada, pelo Presidente da Seccional, a instauração de processo disciplinar. 
 
Sobre o procedimento adotado pelo Presidente da Seccional 
em questão, assinale a afirmativa correta. 

OPTIONS

A) Deveria ter instaurado processo disciplinar quando 
recebeu a missiva anônima. 

B) Não poderia ter instaurado processo disciplinar em 
nenhuma das oportunidades. 

C) Deveria ter instaurado processo disciplinar em qualquer 
uma das oportunidades. 

\textbf{D:CORRECT) Poderia ter instaurado processo disciplinar a partir da publicação da matéria jornalística.}

\subsection{Fundamento}

A resposta está no Código de Ética. artigo 51:

Art. 51. O processo disciplinar instaura-se de ofício ou mediante representação dos interessados, que não pode ser anônima.

\subsection{Algoritmo}

O algoritmo acertou o diploma legal, Código de Ética, e o artigo!

\begin{lstlisting}
        "OAB:2015-18|Q6|ans:D|just:.": {
            "A": [
                4.769202763450931,
                [
                    "6",
                    "urn:lex:br:ordem.advogados.brasil;conselho.federal:codigo.etica:1995-3-1;seq-oab-1art51",
                    "A"
                ]
            ],
            "B": [
                8.732610992128661,
                [
                    "6",
                    "urn:lex:br:ordem.advogados.brasil;conselho.federal:codigo.etica:1995-3-1;seq-oab-1art51",
                    "B"
                ]
            ],
            "C": [
                8.043491345648135,
                [
                    "6",
                    "urn:lex:br:ordem.advogados.brasil;conselho.federal:codigo.etica:1995-3-1;seq-oab-1art51",
                    "C"
                ]
            ],
            "D": [
                10.779468832005046,
                [
                    "6",
                    "urn:lex:br:ordem.advogados.brasil;conselho.federal:codigo.etica:1995-3-1;seq-oab-1art51",
                    "D"
                ]
            ]
        },
        
\end{lstlisting}


\section{Questão 7 do XVIII° Exame Unificado da Ordem (2015)}

\subsection{Enunciado}

Os advogados Márcio, Bruno e Jorge, inscritos nas Seccionais 
do Paraná e de Santa Catarina da Ordem dos Advogados 
resolveram constituir determinada sociedade civil de 
advogados, para atuação na área tributária. A sede da 
sociedade estava localizada em Curitiba. Como os três sócios 
estavam inscritos na Seccional de Santa Catarina, eles 
requereram o registro da sociedade também nessa Seccional. 
Márcio, por outro lado, já fazendo parte da sociedade com 
Bruno e Jorge, requereu, juntamente com seu irmão, 
igualmente advogado, o registro de outra sociedade de 
advogados também na Seccional do Paraná, esta com 
especialização na área tributária. As sociedades não são filiais. 
 
Sobre a hipótese descrita é correto afirmar que a sociedade de 
advogados de Márcio, Bruno e Jorge 
 
OPTIONS

\textbf{A:CORRECT) não poderá ser registrada na seccional de Santa Catarina, 
pois apenas tem sede na Seccional do Paraná. Márcio não 
poderá requerer inscrição em outra sociedade de 
advogados no Paraná. }

B) não poderá ser registrada na seccional de Santa Catarina, 
pois apenas tem sede na Seccional do Paraná. Márcio 
poderá requerer inscrição em outra sociedade de 
advogados no Paraná. 

C) poderá ser registrada na seccional de Santa Catarina, pois 
os três advogados que dela fazem parte estão inscritos na 
Seccional em questão. Márcio não poderá requerer 
inscrição em outra sociedade de advogados no Paraná. 

D) poderá ser registrada na seccional de Santa Catarina, pois 
os três advogados que dela fazem parte estão inscritos na 
Seccional em questão. Márcio poderá requerer inscrição 
em outra sociedade de advogados no Paraná. 

\subsection{Fundamento}

Lei 8906 dispõe em seu art. 15 o seguinte:

 Art. 15.  Os advogados podem reunir-se em sociedade simples de prestação de serviços de advocacia ou constituir sociedade unipessoal de advocacia, na forma disciplinada nesta Lei e no regulamento geral.          (Redação dada pela Lei nº 13.247, de 2016)


§ 1o  A sociedade de advogados e a sociedade unipessoal de advocacia adquirem personalidade jurídica com o registro aprovado dos seus atos constitutivos no Conselho Seccional da OAB em cuja base territorial tiver sede.           (Redação dada pela Lei nº 13.247, de 2016)

§ 2o  Aplica-se à sociedade de advogados e à sociedade unipessoal de advocacia o Código de Ética e Disciplina, no que couber.           (Redação dada pela Lei nº 13.247, de 2016)

§ 3º As procurações devem ser outorgadas individualmente aos advogados e indicar a sociedade de que façam parte.

§ 4o Nenhum advogado pode integrar mais de uma sociedade de advogados, constituir mais de uma sociedade unipessoal de advocacia, ou integrar, simultaneamente, uma sociedade de advogados e uma sociedade unipessoal de advocacia, com sede ou filial na mesma área territorial do respectivo Conselho Seccional.           (Redação dada pela Lei nº 13.247, de 2016)

§ 5o  O ato de constituição de filial deve ser averbado no registro da sociedade e arquivado no Conselho Seccional onde se instalar, ficando os sócios, inclusive o titular da sociedade unipessoal de advocacia, obrigados à inscrição suplementar.              (Redação dada pela Lei nº 13.247, de 2016)

§ 6º Os advogados sócios de uma mesma sociedade profissional não podem representar em juízo clientes de interesses opostos.

§ 7o  A sociedade unipessoal de advocacia pode resultar da concentração por um advogado das quotas de uma sociedade de advogados, independentemente das razões que motivaram tal concentração.             (Incluído pela Lei nº 13.247, de 2016)

\subsection{Algoritmo}

O algoritmo acerto o artigo e a lei.

\begin{lstlisting}
          "OAB:2015-18|Q7|ans:A|just:.": {
            "A": [
                6.928353226093702,
                [
                    "7",
                    "urn:lex:br:federal:lei:1994-07-04;8906art15",
                    "A"
                ]
            ],
            "B": [
                6.928353226093702,
                [
                    "7",
                    "urn:lex:br:federal:lei:1994-07-04;8906art15",
                    "B"
                ]
            ],
            "C": [
                7.1730288556257475,
                [
                    "7",
                    "urn:lex:br:federal:lei:1994-07-04;8906art15",
                    "C"
                ]
            ],
            "D": [
                7.1730288556257475,
                [
                    "7",
                    "urn:lex:br:federal:lei:1994-07-04;8906art15",
                    "D"
                ]
            ]
        },
        
\end{lstlisting}



\section{Questão 8 do XVIII° Exame Unificado da Ordem (2015)}

\subsection{Enunciado}

Gabriela é sócia de uma sociedade de advogados, tendo, no 
exercício de suas atividades profissionais, representado 
judicialmente Júlia. Entretanto, Gabriela, agindo com culpa, 
deixou de praticar ato imprescindível à defesa de Júlia em 
processo judicial, acarretando-lhe danos materiais e morais.  
 
Em uma eventual demanda proposta por Júlia, a fim de ver 
ressarcidos os danos sofridos, deve-se considerar que 

OPTIONS

A) Gabriela e a sociedade de advogados não podem ser 
responsabilizadas civilmente pelos danos, pois, no 
exercício profissional, o advogado apenas responde pelos 
atos que pratica mediante dolo, compreendido por meio 
do binômio consciência e vontade. 

B) a sociedade de advogados não pode ser responsabilizada 
civilmente pelos atos ou omissões praticados 
pessoalmente por Gabriela. Assim, apenas a advogada 
responderá pela sua omissão decorrente de culpa, no 
âmbito da responsabilidade civil e disciplinar. 

C) Gabriela e a sociedade de advogados responderão 
civilmente pela omissão decorrente de culpa, sem prejuízo 
da responsabilidade disciplinar da advogada, cuidando-se 
de hipótese de responsabilidade civil solidária entre 
ambas. 

\textbf{D:CORRECT) Gabriela e a sociedade de advogados podem ser responsabilizadas civilmente pela omissão decorrente de 
culpa. A responsabilidade civil de Gabriela será subsidiária 
à da sociedade e ilimitada pelos danos causados, sem 
prejuízo de sua responsabilidade disciplinar. }

\subsection{Fundamento}

Lei 8906, artigo 17:

Art. 17.  Além da sociedade, o sócio e o titular da sociedade individual de advocacia respondem subsidiária e ilimitadamente pelos danos causados aos clientes por ação ou omissão no exercício da advocacia, sem prejuízo da responsabilidade disciplinar em que possam incorrer.               (Redação dada pela Lei nº 13.247, de 2016)

\subsection{Algoritmo}

O algoritmo acertou a lei e o artigo

\begin{lstlisting}

        "OAB:2015-18|Q8|ans:D|just:.": {
            "A": [
                10.417957132916307,
                [
                    "8",
                    "urn:lex:br:ordem.advogados.brasil;conselho.federal:regulamento.geral:1994-10-16;seq-oab-1art40",
                    "A"
                ]
            ],
            "B": [
                10.795576546669126,
                [
                    "8",
                    "urn:lex:br:ordem.advogados.brasil;conselho.federal:regulamento.geral:1994-10-16;seq-oab-1art40",
                    "B"
                ]
            ],
            "C": [
                9.043843894199348,
                [
                    "8",
                    "urn:lex:br:federal:lei:1994-07-04;8906art17",
                    "C"
                ]
            ],
            "D": [
                7.770359057252578,
                [
                    "8",
                    "urn:lex:br:federal:lei:1994-07-04;8906art17",
                    "D"
                ]
            ]
        },
        
\end{lstlisting}


\bibliographystyle{plain}
\bibliography{references}

\end{document}
